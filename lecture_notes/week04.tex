% ------------------------------------------------------------------
\renewcommand{\thisweek}{MATH327 Week 4}
\renewcommand{\moddate}{Last modified 31 Jan.~2021}
\setcounter{section}{4}
\phantomsection
\addcontentsline{toc}{section}{Week 4: Ideal gases}
\section*{Week 4: Ideal gases}

\subsection{The (non-relativistic) classical ideal gas}
This week we apply the canonical ensemble to investigate non-relativistic, classical, ideal gases.
Using statistical physics we will explore how the large-scale behaviour of such gases emerges from the fundamental features of the particles that it consists of.
The three adjectives above precisely specify these fundamental features: \\[-24 pt]
\begin{itemize}
  \item \textbf{Classical} systems are those for which we can ignore the effects of quantum mechanics.\footnote{In general in physics, `classical' means precisely `non-quantum'.}
        At a technical level, this means we assume we can simultaneously define both the position $\vec x$ and the momentum $\vec p$ of each particle with arbitrary accuracy.
  \item \textbf{Non-relativistic} particles move with speeds small compared to the speed of light, which allows us to ignore small effects due to special relativity.
        The particles are therefore governed by the laws Isaac Newton published all the way back in 1687.
        In particular, the energy of each particle of mass $m$ is
        \begin{equation*}
          E_i = \frac{1}{2m} \vec{p}_i^{\,2}.
        \end{equation*}
  \item \textbf{Ideal} gases are those whose constituent particles don't interact with each other.
        As a result, the total energy of the gas is simply the sum of the energies of the $N$ individual particles,
        \begin{equation}
          E = \frac{1}{2m} \sum_{i = 1}^N \vec{p}_i^{\,2}.
        \end{equation}
\end{itemize}

As usual for the canonical ensemble, we consider the gas enclosed in a box with volume $V = L^3$, which is in thermal equilibrium with a large external thermal reservoir with which the gas can exchange energy.


\TODO{Being written...}

% In order to define the partition function, we need to `quantize' the allowable values of the momenta that each particle can have, so that we have a sum over a countably infinite number of terms that can be enumerated.
% Quantum mechanics explains how this quantization arises, but for the purposes of this module we will just take it as given.
% ------------------------------------------------------------------
