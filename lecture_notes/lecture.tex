% ------------------------------------------------------------------
\documentclass[12 pt]{article}
\pagestyle{plain}
\pagenumbering{arabic}

\setlength{\parindent}{10 mm}
\setlength{\parskip}{10 pt}

\usepackage{amsmath}
\usepackage{amssymb}
\usepackage{graphicx}
\usepackage{verbatim}    % For comment
\usepackage{geometry}
\usepackage[pdftex, pdfstartview={FitH}, pdfnewwindow=true, colorlinks=true, citecolor=blue, filecolor=blue, linkcolor=blue, urlcolor=blue, pdfpagemode=UseNone]{hyperref}

%\usepackage{epsfig}
%\usepackage{amsfonts}
%\usepackage{float}
\usepackage{framed,color}
\usepackage{fancybox}
\usepackage{varwidth}
\definecolor{shadecolor}{rgb}{1,0.8,0.3}
\usepackage[framemethod=tikz]{mdframed}

% Formatting
% The following little hack cancels the change to using paragraphs rather than explicit \medskip
\newcommand{\mysection}[1]{\vspace{-\medskipamount}\section{#1}\vspace{-\medskipamount}}
\newcommand{\mysubsection}[1]{\vspace{-\medskipamount}\subsection{#1}\vspace{-\medskipamount}}
\newcommand{\mutesection}[1]{\vspace{-\medskipamount}\section*{#1}\vspace{-\medskipamount}}
\newcommand{\mutesubsection}[1]{\vspace{-\medskipamount}\subsection*{#1}\vspace{-\medskipamount}}
% ------------------------------------------------------------------



% ------------------------------------------------------------------
% Shortcuts
\newcommand{\Nbb}{\ensuremath{\mathbb N} }
\newcommand{\cO}{\ensuremath{\mathcal O} }
\newcommand{\Rbb}{\ensuremath{\mathbb R} }
\newcommand{\De}{\ensuremath{\Delta} }
\newcommand{\si}{\ensuremath{\sigma} }
\newcommand{\om}{\ensuremath{\omega} }
\newcommand{\Om}{\ensuremath{\Omega} }
\newcommand{\vev}[1]{\ensuremath{\left\langle #1 \right\rangle} }
\newcommand{\TODO}[1]{\textcolor{red}{\textbf{#1}}}
% ------------------------------------------------------------------



% ------------------------------------------------------------------
\begin{document}
\begin{center}
  \textcolor{white}{spacing hack}
  \vfill
  {\LARGE \textbf{MATH327: Statistical Physics}} \\[6 pt]
  \textbf{David Schaich \qquad\qquad\qquad\qquad Spring 2021}
  \vfill
  {\LARGE LECTURE \ NOTES} \\[6 pt]
  Last modified 26 October 2020
  \vfill
\end{center}
\clearpage
% ------------------------------------------------------------------



% ------------------------------------------------------------------
\mutesection{Logistics}
\mutesubsection{Coordinator}
\begin{description}
  \setlength{\itemsep}{1pt}
  \setlength{\parskip}{0pt}
  \setlength{\parsep}{0pt}
  \item[\qquad] Dr David Schaich
  \item[\qquad] Department of Mathematical Sciences
  \item[\qquad] Mathematics Building Room 124 (Theoretical Physics Wing)
  \item[\qquad] \href{http://www.davidschaich.net}{www.davidschaich.net}
  \item[\qquad] \href{mailto:david.schaich@liverpool.ac.uk}{david.schaich@liverpool.ac.uk}
\end{description}

\mutesubsection{Schedule}
\TODO{...}
\begin{comment}
\begin{tabular}{|l|p{12cm}|}
  \hline
  \textbf{Weeks 1--7:}  & We will have \textit{three hours of lecture} and \textit{one tutorial}.
                          There will be question sheets for the tutorials.
                          You will work through part of these questions in your time before the tutorial (non-assessed).
                          Key elements will be discussed in the tutorial. \\[12 pt]
  \hline
  \textbf{Week 8--10:}  & We will have \textit{two hours of lecture} and \textit{two hours Computer Lab}.
                          In the Computer Lab sessions, we will study a statistical phenomenon (see page~\pageref{sec:project}) with a computer experiment using MATLAB.
                          A basic introduction to MATLAB will be provided, but familiarising yourself with MATLAB (if needed) could be beneficial.
                          Support material to get started with MATLAB is provided at the VITAL page for MATH327. \\[12 pt]
  \hline
  \textbf{Week 11--12:} & We will have \textit{three hours of lecture} and \textit{one tutorial}.
                          Both will include exam revision sessions. \\[12 pt]
  \hline
\end{tabular}
\end{comment}
% ------------------------------------------------------------------



% ------------------------------------------------------------------
\mutesubsection{Assessment}
\begin{tabular}{ll}
  30\% & Two homeworks with equal weighting, \TODO{due in ...} \\
  20\% & A two-part computer project \TODO{due in ...} \\
  50\% & Standard examination \\
\end{tabular}

For all assignments, clear and neat presentations of your workings and the logic guiding them will contribute to your mark.

\href{https://www.python.org}{Python} is the default (free) programming language for the computer project, but if you prefer you will also be able to use other languages such as \href{https://www.r-project.org}{R} or even \href{https://matlab.mathworks.com}{MATLAB} (through the University's site license).

\mutesubsection{Resources}
All resources for this module are gathered at its Canvas site.
In addition to these lecture notes, these resources include explanatory videos, sample problems \TODO{and more...}

\textbf{Co-creation:} You can find the \LaTeX\ source for these lecture notes at \\
\centerline{\href{https://github.com/daschaich/MATH327_2021}{github.com/daschaich/MATH327\_2021},}
where you are welcome to report issues and submit pull requests to correct them.

The lists of further reading below provide some optional additional resources that may be helpful.
Hyperlinks lead either to the resource itself or to the corresponding record page in our Library.

\noindent\textbf{These books and lecture notes are roughly at the level of this module:} \\[-24 pt]
\begin{enumerate}
  \item David Tong, \href{https://www.damtp.cam.ac.uk/user/tong/statphys.html}{\textit{Lectures on Statistical Physics}} (2012), \\ www.damtp.cam.ac.uk/user/tong/statphys.html
  \item Daniel V.~Schroeder, \textit{An Introduction to Thermal Physics} (first edition, 2000)
  \item Charles Kittel and Herbert Kroemer, \textit{Thermal Physics} (second edition, 1980)
  \item F.~Reif, \textit{Fundamentals of Statistical and Thermal Physics} (first edition, 1965)
\end{enumerate}

\noindent\textbf{These books are more advanced and more specialized, but can be useful to consult concerning specific questions or topics:} \\[-24 pt]
\begin{enumerate}
  \setcounter{enumi}{4}
  \item Weinan E, Tiejun Li and Eric Vanden-Eijnden, \textit{Applied Stochastic Analysis} (first edition, 2019)
  \item R.~K.~Pathria, \textit{Statistical Mechanics} (second edition, 1996)
  \item Sidney Redner, \textit{A Guide to First-Passage Processes} (first edition, 2007)
  \item Pavel L.~Krapivsky, Sidney Redner and Eli Ben-Naim, \textit{A Kinetic View of Statistical Physics} (first edition, 2010)
  \item Kerson Huang, \textit{Statistical Mechanics} (second edition, 1987)
  \item Michael Plischke and Birger Bergersen, \textit{Equilibrium Statistical Physics} (third edition, 2005)
  \item L.~D.~Landau and E.~M.~Lifshitz, \textit{Statistical Physics, Part 1} (third edition, 1980)
\end{enumerate}

\noindent\textbf{Maple and MATLAB resources:} \\[-24 pt]
\begin{enumerate}
  \setcounter{enumi}{11}
  \item Ian Thompson, \href{https://library.liv.ac.uk/record=b4395758~S8}{\textit{Understanding Maple}} (ebook edition, 2017) \\
        Related videos: \href{https://stream.liv.ac.uk/4k67bdzt}{\textit{Running Maple for the first time}} (stream.liv.ac.uk/4k67bdzt) \\
        \textcolor{white}{Related videos:} \href{https://stream.liv.ac.uk/7pjge23a}{\textit{Configuring Maple}} (stream.liv.ac.uk/7pjge23a)
  \item Stormy Attaway, \textit{MATLAB: A Practical Introduction to Programming and Problem Solving} (third edition, 2013)
  \item B.~Barnes and G.~R.~Fulford, \textit{Mathematical Modelling with Case Studies: Using Maple and MATLAB} (third edition, 2014)
\end{enumerate}

Finally, there is a vast constellation of purely online resources, such as \href{https://en.wikipedia.org/wiki/Statistical_physics}{Wikipedia}.
These are often fine places to \emph{start} learning about a subject, but tend to be terrible places to \emph{stop}.
% ------------------------------------------------------------------



% ------------------------------------------------------------------
\newpage
\tableofcontents
% ------------------------------------------------------------------



% ------------------------------------------------------------------
\begin{comment}
\newpage
\input{ch1_central.tex}

\newpage
\input{ch2_entropy.tex}

\newpage
\input{ch3_canonical.tex}

\newpage
\input{ch4_ideal.tex}

\newpage
\input{ch5_cycle.tex}

\newpage
\input{ch6_grand.tex}

\newpage
\input{ch8_quantum.tex}

\newpage
\input{ch9_transitions.tex}

\newpage
\input{appendices.tex}

\newpage
\input{appendices_hw.tex}
\end{comment}

\end{document}
% ------------------------------------------------------------------
