% ------------------------------------------------------------------
\renewcommand{\thisweek}{MATH327 Week 8}
\renewcommand{\moddate}{Last modified 17 Apr.~2021}
\setcounter{section}{8}
\setcounter{subsection}{0}
\phantomsection
\addcontentsline{toc}{section}{Week 8: Quantum gases of bosons}
\section*{Week 8: Quantum gases of bosons}
\subsection{The photon gas}
Last week we derived the grand-canonical partition function (\eq{eq:partfunc_BE}) that defines quantum Bose--Einstein statistics for systems of non-interacting bosons,
\begin{equation*}
  \ZBE(\be, \mu) = \prod_{\ell = 0}^L \frac{1}{1 - e^{-\be (E_{\ell} - \mu)}}.
\end{equation*}
This expression results from summing over the possible occupation numbers $n_{\ell} \in \Nbb_0$ for each energy level $\cE_{\ell}$ with energy $E_{\ell}$.
The corresponding grand-canonical potential is
\begin{equation*}
  \Phi_{\text{BE}} = -T \log Z_g = T \sum_{\ell = 0}^L \log\left[1 - e^{-\be (E_{\ell} - \mu)}\right],
\end{equation*}
from which we can determine the large-scale properties of the system, including its average internal energy $\vev{E}$, average particle number $\vev{N}$, entropy $S$, and pressure $P$.

To do so, we have to specify the energy levels of the particles that compose the system of interest, and the degeneracies of those energy levels.
One example of this that we have already seen is the analysis of non-relativistic ideal gas particles in \secref{sec:regulate}.
For a single particle with mass $m$ in a volume $V = L^3$, we determined the quantized energies
\begin{equation}
  \label{eq:nonrel_energy}
  E(k_x, k_y, k_z) = \frac{\hbar^2 \pi^2}{2mL^2}\left(k_x^2 + k_y^2 + k_z^2\right),
\end{equation}
where the integers $k_{x, y, z}$ specify the possible momenta of the particle,
\begin{align*}
  \vec p & = (p_x, p_y, p_z) = \hbar \frac{\pi}{L} (k_x, k_y, k_z) &
  k_{x, y, z} = 1, 2, \cdots.
\end{align*}
(For technical reasons, quantum mechanics requires $k_{x, y, z} \geq 1$, leading us to adjust our ansatz compared to \eq{eq:quant_mom}.)
This system has a unique ground state $\cE_0$ with $\vec k = (1, 1, 1)$ and energy $E_0 = \frac{3}{2} \frac{\hbar^2 \pi^2}{mL^2}$.
The next three energy levels are degenerate, with energy $3 \frac{\hbar^2 \pi^2}{mL^2}$ corresponding to $\vec k = (2, 1, 1)$ and permutations, followed by another three degenerate energy levels with energy $\frac{9}{2} \frac{\hbar^2 \pi^2}{mL^2}$ corresponding to $\vec k = (2, 2, 1)$ and permutations.

This week we will build on that experience to consider a gas of \textit{photons}, massless bosonic quantum particles of light.
For our purposes, with no prior knowledge of particle physics, we can define photons simply by specifying two relevant details of their energy levels.
First, a photon's energy is proportional to the magnitude of its momentum, %with $m = 0$ for massless photons, \eq{eq:nonrel_energy} is clearly problematic.
\begin{equation*}
  \Eph(p) = c \sqrt{p_x^2 + p_y^2 + p_z^2} \equiv c p.
\end{equation*}
Here the speed of light $c$ is really just a unit conversion factor that we could set to $c = 1$ by using appropriate units.
Second, for each momentum $\vec p$, a photon has two degenerate energy levels with the same energy $E(p)$. % TODO: Could mention polarization...

In a volume $V = L^3$, only the same discrete momenta as above are allowed,
\begin{align*}
  p & = \hbar \frac{\pi}{L} \sqrt{k_x^2 + k_y^2 + k_z^2} \equiv \hbar \frac{\pi}{L} k &
  k_{x, y, z} = 1, 2, \cdots,
\end{align*}
so that the quantized photon energies are
\begin{equation}
  \label{eq:photon_Ek}
  \Eph(k) = \hbar c \frac{\pi}{L} k.
\end{equation}
It is conventional to use the speed of light to work with photons in terms of their wavelength \la and angular frequency $\om = 2\pi f$ (not to be confused with generic micro-states $\om_i$), given the relation
\begin{equation*}
  c = \frac{\la \om}{2\pi}.
\end{equation*}
Just like the momenta, the wavelengths \la are also quantized in volume $V = L^3$,
\begin{equation*}
  \la = \frac{2L}{k} \qquad \Lra \qquad c = \frac{\om}{\frac{\pi}{L} k},
\end{equation*}
and we can rewrite \eq{eq:photon_Ek} as
\begin{equation}
  \label{eq:photon_omega}
  \Eph(\om) = \hbar \om.
\end{equation}
Low (\textit{infrared}) frequencies correspond to small energies and long wavelengths, while high (\textit{ultraviolet}) frequencies correspond to large energies and short wavelengths.

We are now ready to write down the grand-canonical potential for a photon gas:
\begin{equation*}
  \Phi_{\text{ph}} = T \sum_{\ell = 0}^L \log\left[1 - e^{-\be (E_{\ell} - \mu)}\right] = 2T \sum_{\vec k} \log\left[1 - e^{-\be (\Eph(k) - \mu)}\right],
\end{equation*}
where the factor of $2$ in the final expression accounts for the doubly degenerate energy levels.
We can simplify this expression by appreciating that photons are easy to create and destroy.
Every time a light is switched on, it begins emitting a constant flood of photons (with wavelengths of several hundred nanometres).
Food in a microwave oven gets hot by absorbing many lower-energy photons (with longer wavelengths around $12$~centimetres).
In both cases an enormous number of photons is required to make even a small change in energy, so that \eq{eq:mu_E} implies the chemical potential of a photon gas must be negligible,
\begin{equation*}
  \mu = \left.\pderiv{E}{N}\right|_S \approx 0 \qquad \Lra \qquad \Phi_{\text{ph}} \approx 2T \sum_{\vec k} \log\left[1 - e^{-\be \Eph(k)}\right].
\end{equation*}

Another simplification comes from considering the photon gas in a large volume, so that we can approximate the sum over discrete integer $k_{x, y, z}$ by integrals over continuous real $\khat_{x, y, z}$,
\begin{equation*}
  \Phi_{\text{ph}} \approx 2T \int d\khat_x d\khat_y d\khat_z \log\left[1 - e^{-\be \Eph(\khat)}\right].
\end{equation*}
Since the energy $\Eph(\khat)$ depends only on the magnitude $\khat$, we can profit from converting to spherical coordinates.
When we do so, we have to keep in mind that $k_{x, y, z} > 0$ corresponds only to the positive octant of thee sphere,
\begin{equation*}
  \int_0^{\infty} d\khat_x \int_0^{\infty} d\khat_y \int_0^{\infty} d\khat_z = \int_0^{\infty} d\khat \; \khat^2 \int_0^{\pi / 2} d\theta \; \sin\theta \int_0^{\pi / 2} d\phi = \frac{\pi}{2} \int_0^{\infty} d\khat \; \khat^2,
\end{equation*}
so that
\begin{equation*}
  \Phi_{\text{ph}} \approx \pi T \int_0^{\infty} d\khat \; \khat^2 \log\left[1 - e^{-\be \Eph(\khat)}\right].
\end{equation*}
We can finally change variables to integrate over the photon angular frequency $\om = c \frac{\pi}{L} k$, with $\Eph = \hbar \om$, to find
\begin{align}
  \Phi_{\text{ph}} & \approx \pi T \left(\frac{L}{\pi c}\right)^3 \int_0^{\infty} d\om \; \om^2 \log\left[1 - e^{-\be \hbar \om}\right] \cr
                   & = \frac{VT}{c^3 \pi^2} \int_0^{\infty} d\om \; \om^2 \log\left[1 - e^{-\be \hbar \om}\right], \label{eq:photon_grand}
\end{align}
recognizing $L^3 = V$.
With this grand-canonical potential derived, we just need to take the appropriate derivatives to determine the thermodynamics and equation of state for the photon gas.
% ------------------------------------------------------------------



% ------------------------------------------------------------------
\subsection{The sun and the void}
We are now ready to analyze the average internal energy from the grand-canonical potential for a photon gas, \eq{eq:photon_grand}.
With $\mu = 0$, \eq{eq:E_grand} from week~6 becomes
\begin{equation*}
  \vev{E}_{\text{ph}} = -T^2 \pderiv{}{T} \left[\frac{\Phi_{\text{ph}}}{T}\right] = \pderiv{}{\be} \left[\be \Phi_{\text{ph}}\right].
\end{equation*}
To begin, we will consider the energy density expressed as an integral over photon frequencies,
\begin{equation*}
  \frac{\vev{E}_{\text{ph}}}{V} = \int_0^{\infty} P(\om) \; d\om,
\end{equation*}
where the function $P(\om)$ is known as the \textit{spectral density}, or simply the \textit{spectrum}.
What is the spectrum for a photon gas?
\begin{mdframed}
  $\displaystyle \frac{\vev{E}_{\text{ph}}}{V} = \frac{1}{c^3 \pi^2} \int_0^{\infty} d\om \; \om^2 \pderiv{}{\be} \log\left[1 - e^{-\be \hbar \om}\right] = $ \\[100 pt]
\end{mdframed}

You should find
\begin{equation}
  \label{eq:Planck_omega}
  P(\om) = \left(\frac{\hbar}{c^3 \pi^2}\right) \frac{\om^3}{e^{\be \hbar \om} - 1},
\end{equation}
which is known as the Planck spectrum, named after Max Planck.
The Planck spectrum is plotted in the figure below, which comes from \href{https://commons.wikimedia.org/wiki/File:Black_body.svg}{Wikimedia Commons}.

\begin{center}\includegraphics[width=\textwidth]{figs/week08_spectrum.pdf}\end{center}

In this plot the horizontal axis uses the wavelength $\la = 2\pi c / \om$.
Changing variables in your work above, what is Planck spectrum $P(\la)$ as a function of wavelength?
\begin{mdframed}
  $\displaystyle \frac{\vev{E}_{\text{ph}}}{V} = \frac{\hbar}{c^3 \pi^2} \int_0^{\infty} \frac{\om^3}{e^{\be \hbar \om} - 1} \; d\om = $ \\[120 pt] % WARNING: FORMATTING BY HAND
\end{mdframed}

You should find
\begin{equation}
  \label{eq:Planck_la}
  P(\la) = \left(\frac{16\pi^2 \hbar c}{\la^5}\right) \frac{1}{e^{2\pi\be \hbar c / \la} - 1},
\end{equation}
which is plotted\footnote{The plot divides our $P(\la)$ by $4\pi$~steradian to consider the spectrum per unit of solid angle.} for three temperatures $T = 1 / \be$ in the figure above.
Considering first the high-energy ultraviolet (UV) limit of small wavelengths $\la$, we can see from \eq{eq:Planck_la} that $P(\la)$ is exponentially suppressed, which overwhelms the diverging factor $\propto 1 / \la^5$ in parentheses.

In the low-energy infrared limit, the large $\la$ has the same effect that a large temperature ($\be \ll 1$) would have: $e^{2\pi\be \hbar c / \la} - 1 \approx 2\pi\be \hbar c / \la$ and
\begin{equation*}
  P(\la) \approx \left(\frac{16\pi^2 \hbar c}{\la^5}\right) \frac{\la}{2\pi\be \hbar c} = \frac{8\pi T}{\la^4}.
\end{equation*}
The connection to large temperatures indicates that this is the result classical statistics would predict for the energy spectrum of light.
It is known as the Rayleigh--Jeans spectrum (named after \href{https://en.wikipedia.org/wiki/John_William_Strutt,_3rd_Baron_Rayleigh}{the third Baron Rayleigh} and \href{https://en.wikipedia.org/wiki/James_Jeans}{James Jeans}), and clearly cannot be correct in the ultraviolet limit $\la \to 0$, where it predicts short-wavelength light would possess a diverging amount of energy.
This classical result is known as the \textit{ultraviolet catastrophe}, and Planck's (heuristic) solution to it was one of the first steps towards quantum physics.


As the temperature increases, the maximum of the Planck spectrum moves to shorter wavelengths and correspondingly larger energies.
The fact that the peak of the spectrum for $T \approx 5000$~K falls around the wavelengths of visible light (roughly $400$--$700$~nm) is not a coincidence.
As shown in the figure below, the 

\begin{center}\includegraphics[width=\textwidth]{figs/week08_sun.pdf}\end{center}



\TODO{Being written...}
% ------------------------------------------------------------------



% ------------------------------------------------------------------
\newpage
\subsection{Photon gas equation of state}
\TODO{Being written...}
% ------------------------------------------------------------------



% ------------------------------------------------------------------
\newpage
\subsection{Bose--Einstein condensation}
\TODO{Being written...} % BEC now routinely produced in hundreds of labs around the world, including as tool for quantum computing (e.g., arXiv:2003.08945)...
% ------------------------------------------------------------------
