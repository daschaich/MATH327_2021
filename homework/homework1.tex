% ------------------------------------------------------------------
\documentclass[12 pt]{article} % A4 paper set by geometry package below
\pagenumbering{arabic}
\setlength{\parindent}{10 mm}
\setlength{\parskip}{12 pt}

% Nimbus Sans font should be reasonably legible
\usepackage{helvet}
\renewcommand{\familydefault}{\sfdefault}
\usepackage[T1]{fontenc}  % Without this \textsterling produces $

% Section header spacing
\usepackage{titlesec}
\titlespacing\section{0pt}{12pt plus 4pt minus 2pt}{0pt plus 2pt minus 2pt}
\titlespacing\subsection{0pt}{12pt plus 4pt minus 2pt}{0pt plus 2pt minus 2pt}
\titlespacing\subsubsection{0pt}{12pt plus 4pt minus 2pt}{0pt plus 2pt minus 2pt}

\usepackage{amsmath}
\usepackage{amssymb}
\usepackage{graphicx}
\usepackage{verbatim}    % For comment
\usepackage[paper=a4paper, marginparwidth=0 cm, marginparsep=0 cm, top=2.5 cm, bottom=2.5 cm, left=3 cm, right=3 cm, includemp]{geometry}
\usepackage[pdftex, pdfstartview={FitH}, pdfnewwindow=true, colorlinks=true, citecolor=blue, filecolor=blue, linkcolor=blue, urlcolor=blue, pdfpagemode=UseNone]{hyperref}

\usepackage{framed,color}
\usepackage{fancybox}
\usepackage{varwidth}
\definecolor{shadecolor}{rgb}{1,0.8,0.3}
\usepackage[framemethod=tikz]{mdframed}

% Put module code and last-modified date in footer
\usepackage{fancyhdr}
\pagestyle{fancy}
\fancyhf{}
\renewcommand{\headrulewidth}{0pt}
\cfoot{{\small \thisweek}\hfill \thepage\hfill {\small \moddate}}

% Hopefully address Canvas complaints about pdf tagging
%\usepackage[tagged]{accessibility}
% ------------------------------------------------------------------



% ------------------------------------------------------------------
% Shortcuts
\newcommand{\be}{\ensuremath{\beta} }
\newcommand{\De}{\ensuremath{\Delta} }
\newcommand{\si}{\ensuremath{\sigma} }
\newcommand{\vdr}{\ensuremath{v_{\mathrm{dr}}} }
\newcommand{\vev}[1]{\ensuremath{\left\langle #1 \right\rangle} }
\newcommand{\pderiv}[2]{\ensuremath{\frac{\partial #1}{\partial #2}} }
\newcommand{\showmarks}[1]{\rightline{\texttt{[#1 marks]}}} % \showmarks needs to follow a blank line!
% ------------------------------------------------------------------



% ------------------------------------------------------------------
\begin{document}
\newcommand{\thisweek}{MATH327 Homework 1}
\newcommand{\moddate}{Last modified 5 Mar.~2021}
\begin{center}
  {\Large \textbf{MATH327: Statistical Physics, Spring 2021}} \\[12 pt]
  {\Large \textbf{Homework assignment 1}} \\[24 pt]
\end{center}

\section*{Instructions}
Complete all four questions below and submit your solutions by file upload \href{https://liverpool.instructure.com/courses/19478/assignments/89667}{on Canvas}.
Clear and neat presentations of your workings and the logic behind them will contribute to your mark.
This assignment is \textbf{due by 23:59 on Tuesday, 16 March}.
Anonymous marking is turned on, and I will aim to return feedback during the term break.
% ------------------------------------------------------------------



% ------------------------------------------------------------------
\section*{Question 1: Probability intervals}
Consider a random walk that consists of $N$ steps, with the length $x_i$ of each step randomly drawn from the probability distribution
\begin{equation*}
  p_1(x) = \left\{\begin{array}{ll}\frac{3}{4} x (2 - x) & \mbox{for } 0 \leq x < 2 \\
                                   0                     & \mbox{otherwise}\end{array}\right. .
\end{equation*}
What is the most likely step length?
What are the mean step length $\mu$ and its standard deviation $\si$?

\showmarks{3}

After many steps are taken, $N \gg 1$, the final position of the walker is
\begin{equation*}
  X = \sum_{i = 1}^N x_i.
\end{equation*}
What is the probability distribution $p(X)$ for this final position $X$ from the central limit theorem?
What are the corresponding expectation value $\vev{X}$ and standard deviation $\ell_2 = \sqrt{\vev{X^2} - \vev{X}^2}$ as functions of $N$?

\showmarks{3}

After $N = 100$ steps, what are the intervals
\begin{equation*}
  \vev{X} - \De \leq X \leq \vev{X} + \De
\end{equation*}
within which the walker can be found with 68\%, 96\% and 99\% probabilities?

Hint: The following values of the error function may be useful:
\begin{align*}
  \mathrm{erf}(u) = \frac{1}{\sqrt{\pi}} \int_{-u}^u e^{-x^2} \; dx = \left\{\begin{array}{ll}0.68 & \mbox{for } u \approx 0.7032 \\
                                                                                              0.96 & \mbox{for } u \approx 1.4522 \\
                                                                                              0.99 & \mbox{for } u \approx 1.8214\end{array}\right.
\end{align*}

\showmarks{6}
% ------------------------------------------------------------------



% ------------------------------------------------------------------
\section*{Question 2: Drift and diffusion}
There has been an oil spill $1~\mathrm{km}$ away from a \href{https://en.wikipedia.org/wiki/Marine_protected_area}{marine protected area} (MPA).
$5000$ tonnes of oil are in the water.
The motion of each oil droplet can be modeled as a one-dimensional random walk towards (or away from) the MPA, with diffusion constant $D = 15~\mathrm{cm}/\sqrt{\mathrm{min}}$.
You have been called on to estimate how much time is available for the government to take action to protect the MPA.
If the oil has a drift velocity $\vdr = 50~\mathrm{cm}/\mathrm{min}$ towards the MPA, how long will it take until $100$ tonnes of oil is inside the MPA?

Hint: The error function in the previous question may be useful.

\showmarks{12}

If the oil were washing up on a shore instead of entering an MPA, the mathematics would be significantly more complicated, because each droplet's random walk would \textit{stop} once it reached the shore and left the water.
This is known as a \textit{first-passage process}.
Without attempting this more complicated mathematics, explain whether it will take more time, less time or the same amount time for $100$ tonnes of oil to wash up on shore, compared to the MPA case considered above, with everything else the same.

\showmarks{3}
% ------------------------------------------------------------------



% ------------------------------------------------------------------
\vfill
\section*{Question 3: Heat capacity}
Starting from the average internal energy for the canonical ensemble,
\begin{equation*}
  \vev{E} = \frac{1}{Z} \sum_{i = 1}^M E_i \; e^{-\be E_i},
\end{equation*}
derive a relation between the heat capacity
\begin{equation*}
  c_v = \pderiv{}{T} \vev{E}
\end{equation*}
and the quantity $\vev{\left(E - \vev{E}\right)^2}$.

\showmarks{6}

What condition(s) must the micro-state energies $E_i$ satisfy in order for the heat capacity to vanish, $c_v = 0$?

\showmarks{2}
% ------------------------------------------------------------------



% ------------------------------------------------------------------
\newpage
\section*{Question 4: Indistinguishable spins}
The Helmholtz free energy for $N \gg 1$ indistinguishable spins is
\begin{equation*}
  F_I(\be) = -NH - \frac{\log\left[1 - e^{-2(N + 1) \be H}\right]}{\be} + \frac{\log\left[1 - e^{-2 \be H}\right]}{\be}.
\end{equation*}
What are the corresponding internal energy $\vev{E}_I$ and entropy $S_I$?

\showmarks{5}

What are the two leading terms in each low-temperature ($e^{-2\be H} \ll 1$) expansion of $\vev{E}_I$ and $S_I$?

\showmarks{5}

What is the leading term in the high-temperature ($\be H \ll 1$) expansion of $\vev{E}_I$?
What are the two leading terms in the high-temperature expansion of $S_I$?

\showmarks{5}
% ------------------------------------------------------------------



% ------------------------------------------------------------------
\end{document}
% ------------------------------------------------------------------
