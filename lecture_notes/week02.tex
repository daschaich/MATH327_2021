% ------------------------------------------------------------------
\renewcommand{\thisweek}{MATH327 Week 2}
\renewcommand{\moddate}{Last modified 2 Feb.~2021}
\setcounter{section}{2}
\setcounter{subsection}{0}
\phantomsection
\addcontentsline{toc}{section}{Week 2: Micro-canonical ensemble}
\section*{Week 2: Micro-canonical ensemble}

\subsection{First law of thermodynamics}
We begin this week by developing the concept of \textit{statistical ensembles} (introduced by \href{https://en.wikipedia.org/wiki/Josiah_Willard_Gibbs}{J.\ Willard Gibbs} in the early 1900s), building on the probability foundations we laid last week.
As forecast last week, we will be interested in `experiments' that simply allow a collection of degrees of freedom to evolve in time, subject to certain constraints.
At a given time $t_1$, the disposition of these degrees of freedom defines the state $\om_1$ of the system.
To consider a couple of examples, what would be a representative state for a system of $8$ \textit{spins} (arrows that can point either up or down) evenly spaced along a line?
What information would characterize the state of $N$ hydrogen (H$_2$) molecules in a container?
\begin{mdframed}
  \ \\[100 pt]
\end{mdframed}

At a different time $t_2$, the system's state $\om_2$ is most likely different from $\om_1$.
However, there are some \textit{measurements} we can perform on these states that do not change as the system evolves in time.
These measurements define \textit{conserved quantities}, an important example of which is the 

\TODO{Being written...}
% ------------------------------------------------------------------
