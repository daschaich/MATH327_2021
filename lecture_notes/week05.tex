% ------------------------------------------------------------------
\renewcommand{\thisweek}{MATH327 Week 5}
\renewcommand{\moddate}{Last modified 26 Feb.~2021}
\setcounter{section}{5}
\setcounter{subsection}{0}
\phantomsection
\addcontentsline{toc}{section}{Week 5: Thermodynamic cycles}
\section*{Week 5: Thermodynamic cycles}
\subsection{Work, pressure and force}
Last week we defined the pressure in the canonical ensemble as the thermodynamic response of the internal energy to an adiabatic change in the volume (\eq{eq:pressure}).
At the same time, we motivated this definition by thinking about `squeezing' the system---exerting a force on it---which suggests a connection between pressure and force.
Here we make this connection explicit by considering how the energy of an object changes when a force acts on it.

\begin{shaded}
  Consider an object at position $\vec r = (x, y, z)$, and suppose it is displaced by a vector $d\vec r$ due to a force $\vec F(\vec r)$.
  The \textbf{work} done by this force is defined to be the resulting change in the energy of the object.
  Infinitesimally, $dW = dE = \vec F\cdot d\vec r$, which generalizes to the line integral $W = \De E = \int \vec F(r)\cdot d\vec r$.
\end{shaded}

A famous example is an object falling due to the force of the Earth's gravity.
That force is $\vec F = (0, 0, -mg)$, where $m$ is the mass of the object, $g \approx 9.8~\mathrm{m}/\mathrm{s}^2$ (metres per second per second) is the strength of gravity near the surface of the Earth, and the negative sign indicates that the gravitational force is directed downward.
The object starts from rest, with initial (kinetic) energy $E_0 = 0$, and falls downward (parallel to $\vec F$) from a height $h$.
Its final energy $E_f$ upon hitting the ground comes from the work done by the Earth's gravity:
\begin{align*}
  W & = \int \vec F(r)\cdot d\vec r = -mg \int_h^0 dz = mgh > 0 \\
  E_f & = E_0 + \De E = 0 + W = mgh = \frac{p_z^2}{2m} \qquad \lra \qquad p_z = m\sqrt{2gh},
\end{align*}
where $\vec p = (p_x, p_y, p_z)$ is the momentum we considered last week (\eq{eq:momentum}).

To connect the concept of work to the pressure of a statistical system described by the canonical ensemble, let's consider the setup shown below (copied from Schroeder's \textit{Introduction to Thermal Physics}).
Here we have an gas in a container of volume $V$, with one wall of that container being a piston that we can displace by applying a force $F$.
Let's demand that this process is adiabatic---it does not change the entropy of the gas.
The displacement $\De x$ shown in the figure reduces the volume of the gas, by $\De V = -A\De x < 0$ where $A$ is the area of the piston.
Since the force $F$ is parallel to the piston's displacement $\De x$, it does positive work $W = F\De x > 0$.
Therefore the energy of the gas increases, at the same time as its volume decreases adiabatically, so from \eq{eq:pressure} we have
\begin{equation}
  P = -\left. \pderiv{}{V} \vev{E}\right|_S = -\frac{W}{\De V} = \frac{F\De x}{A\De x} = \frac{F}{A}
\end{equation}
This identifies the pressure of a gas in a container as the force per unit area that the gas exerts on the container wall, reassuringly consistent with our everyday experiences.

\begin{center}
  \includegraphics[width=0.7\textwidth]{figs/week05_piston.pdf}
\end{center}

Rearranging the expressions above, we can obtain an expression for the work \textit{put into} the gas by its surroundings (that is, by the external force applied to move the piston).
Still assuming an adiabatic (constant-entropy) process, this input work must match the increase in the gas's average internal energy,
\begin{align*}
  W_{\text{in}} & = \De \vev{E} = -P \De V & & \mbox{for constant entropy.}
\end{align*}
If the entropy is allowed to change, this relation between pressure and work will still hold---as we will see in the next section, we will simply no longer be able to identify the work as the total change in the average internal energy:
\begin{align}
  \label{eq:work_constP}
  W_{\text{in}} & = -P \De V & & \mbox{more generally.}
\end{align}
Later we will be interested in using the gas as a thermodynamic engine that \textit{does work on} its surroundings.
This removes energy from the gas, reversing the negative sign above, $W_{\text{done}} = -W_{\text{in}}$.

Of course, as we change the volume of the gas, the pressure itself will change as described by the gas's equation of state---such as the ideal gas law, \eq{eq:ideal_gas_law}.
With the equation of state providing an expression $P(V)$ for the pressure as a function of the volume, \eq{eq:work_constP} generalizes to
\begin{equation}
  \label{eq:work}
  W_{\text{in}} = -\int_{V_0}^{V_f} P(V) dV.
\end{equation}
% ------------------------------------------------------------------



% ------------------------------------------------------------------
\subsection{Heat and entropy}
Now let's switch things up by changing the temperature $T$ of an ideal gas while keeping its volume constant.
(As always for the canonical ensemble, the number of particles $N$ is also constant.)
Since the volume is constant, \eq{eq:work} indicates that no work is done, $W = 0$.
Even so, from \eq{eq:ideal_energy} we have $\vev{E} = \frac{3}{2} NT$ and can see that the average internal energy still changes,
\begin{equation}
  d\vev{E} = \frac{3}{2} N dT.
\end{equation}

How does this change the internal energy $\vev{E}$ and the entropy $S$?


% TODO: Isothermal slow vs. adiabatic fast...
\TODO{Being written...}
% ------------------------------------------------------------------



% ------------------------------------------------------------------
\newpage % TODO: Placeholder...
\subsection{Thermodynamic cycles}
\TODO{Being written...}
% ------------------------------------------------------------------



% ------------------------------------------------------------------
\newpage % TODO: Placeholder...
\subsection{The Carnot cycle}
\TODO{Being written...}
% ------------------------------------------------------------------
