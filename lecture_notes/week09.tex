% ------------------------------------------------------------------
\renewcommand{\thisweek}{MATH327 Week 9}
\renewcommand{\moddate}{Last modified 21 Apr.~2021}
\setcounter{section}{9}
\setcounter{subsection}{0}
\phantomsection
\addcontentsline{toc}{section}{Week 9: Quantum gases of fermions}
\section*{Week 9: Quantum gases of fermions}
\subsection{Non-relativistic ideal fermion gas}
This week we wrap up our applications of the grand-canonical ensemble to investigate ideal gases of non-interacting particles.
We again take the quantum statistical approach of defining micro-states by summing over the possible occupation numbers $n_{\ell}$ for each energy level $\cE_{\ell}$ with energy $E_{\ell}$.
In contrast to the bosonic case considered last week, we now focus on quantum gases of fermions, where the only possible occupation numbers are $n_{\ell} = 0$ and $1$, since the Pauli exclusion principle prevents multiple identical fermions from occupying the same energy level.

In \secref{sec:fermi} we derived the grand-canonical partition function (\eq{eq:partfunc_FD}) that defines quantum Fermi--Dirac statistics for such systems of non-interacting fermions,
\begin{equation*}
  \ZFD(\be, \mu) = \prod_{\ell = 0}^L \left[1 + e^{-\be (E_{\ell} - \mu)}\right],
\end{equation*}
for inverse temperature $\be = 1 / T$ and chemical potential $\mu$.
From the corresponding grand-canonical potential,
\begin{equation*}
  \Phi_{\text{FD}} = -T \sum_{\ell = 0}^L \log\left[1 + e^{-\be (E_{\ell} - \mu)}\right]
\end{equation*}
we can determine the large-scale properties of the system, including its average internal energy $\vev{E}$, average particle number $\vev{N}$, entropy $S$, and pressure $P$, along with the equation of state relating these quantities.

A concrete calculation requires specifying the energy levels of the particles that compose the gas, and the degeneracies of those energy levels.
Let's begin this week by considering non-relativistic particles of the sort we previously analyzed in \secref{sec:regulate}.
In a volume $V = L^3$, the energy levels are defined by the non-zero quantized energies
\begin{align*}
  E(k) & = \frac{p^2}{2m} = \frac{\hbar^2 \pi^2}{2mL^2}\left(k_x^2 + k_y^2 + k_z^2\right) &
  k_{x, y, z} & = 1, 2, \cdots.
\end{align*}
In addition to the usual degeneracies coming from permutations of $(k_x, k_y, k_z)$ that we discussed in previous weeks, for each distinct $\vec k$ typical fermions such as electrons have two degenerate energy levels with the same energy.
This factor of $2$ has a different origin compared to the double degeneracy discussed for photons last week.
Rather than worry about the physical origins of this behaviour, in both cases we simply incorporate this given information into our ansatz. % TODO: Could mention spin...

The grand-canonical potential for an ideal gas of non-relativistic fermions is therefore
\begin{equation*}
  \Phi_{\text{f}} = T \sum_{\ell = 0}^L \log\left[1 + e^{-\be (E_{\ell} - \mu)}\right] = 2T \sum_{\vec k} \log\left[1 + \exp\left(-\frac{\hbar^2 \pi^2 k^2}{2mL^2 T} + \frac{\mu}{T}\right)\right].
\end{equation*}
We can again proceed by considering the gas in a large volume and approximating the sum over discrete integer $k_{x, y, z}$ by integrals over continuous real $\khat_{x, y, z}$:
\begin{equation*}
  \Phi_{\text{f}} \approx 2T \int d\khat_x d\khat_y d\khat_z \log\left[1 + \exp\left(-\frac{\hbar^2 \pi^2 \khat^2}{2mL^2 T} + \frac{\mu}{T}\right)\right].
\end{equation*}
Converting to spherical coordinates and carrying out the angular integrations over the $\frac{\pi}{2}$ solid angle of the octant of the sphere with $k_{x, y, z} > 0$, we have
\begin{equation*}
  \Phi_{\text{f}} \approx \pi T \int_0^{\infty} d\khat \; \khat^2 \log\left[1 + \exp\left(-\frac{\hbar^2 \pi^2 \khat^2}{2mL^2 T} + \frac{\mu}{T}\right)\right].
\end{equation*}
In the same spirit as the change of variables we carried out last week, to integrate over photon frequencies $\om = \Eph / \hbar$, we will now change variables to integrate over the fermion energy:
\begin{align*}
  E = \frac{\hbar^2 \pi^2}{2mL^2}\khat^2 \quad \lra \quad \khat & = \frac{L\sqrt{m}}{\pi \hbar} \sqrt{2E} \cr
                                                         d\khat & = \frac{L\sqrt{m}}{\pi \hbar} \frac{dE}{\sqrt{2E}}.
\end{align*}
Plugging this in produces
\begin{align}
  \Phi_{\text{f}} & \approx \pi T \left(\frac{L^3 m^{3 / 2}}{\pi^3 \hbar^3}\right) \int_0^{\infty} dE \frac{2E}{\sqrt{2E}} \log\left[1 + e^{-\be(E - \mu)}\right] \cr
                  & = \frac{\sqrt{2 m^3} VT}{\pi^2 \hbar^3} \int_0^{\infty} dE \sqrt{E} \log\left[1 + e^{-\be(E - \mu)}\right]. \label{eq:fermi_grand}
\end{align}
recognizing $L^3 = V$.

With this grand-canonical potential derived, we just need to take the appropriate derivatives to determine the thermodynamics and equation of state for non-relativistic fermions.
When doing so, we'll focus on the low-temperature regime where we expect quantum Fermi--Dirac statistics to differ significantly from the classical case we considered back in \secref{sec:regulate}.
As we saw in \secref{sec:quantum_classical}, at high temperatures those classical results provide a good approximation to the true quantum physics.
% TODO: Low temperatures provide a significant simplification, which is why we only consider that regime this week
% ------------------------------------------------------------------



% ------------------------------------------------------------------
\subsection{Low-temperature equation of state}
Rather than the average internal energy, it will prove profitable to first analyze the average particle number from the grand-canonical potential for non-relativistic fermions, \eq{eq:fermi_grand}.


\TODO{...}

\begin{equation*}
  n(E) \approx \left\{\begin{array}{l}1 \qquad \mbox{for } E \leq u < 1 \\
                                      0 \qquad \mbox{otherwise}\end{array}\right. .
\end{equation*}

\TODO{reproduce classical ideal gas law...}

% white dwarfs that accrete matter from a companion giant star, leading to a steadily increasing mass. As the white dwarf's mass approaches the Chandrasekhar limit, its central density increases, and, as a result of compressional heating, its temperature also increases. This eventually ignites nuclear fusion reactions, leading to an immediate carbon detonation, which disrupts the star and causes the supernova
% named after \href{https://en.wikipedia.org/wiki/Subrahmanyan_Chandrasekhar}{Subrahmanyan Chandrasekhar}
\TODO{Being written...}
% ------------------------------------------------------------------


% ------------------------------------------------------------------
\newpage
\subsection{Equation of state\TODO{...}}

\TODO{Being written...}
% ------------------------------------------------------------------
