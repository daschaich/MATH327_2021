% ------------------------------------------------------------------
\documentclass[12 pt]{article} % A4 paper set by geometry package below
\pagenumbering{arabic}
\setlength{\parindent}{10 mm}
\setlength{\parskip}{12 pt}

% Nimbus Sans font should be reasonably legible
\usepackage{helvet}
\renewcommand{\familydefault}{\sfdefault}
\usepackage[T1]{fontenc}  % Without this \textsterling produces $

% Section header spacing
\usepackage{titlesec}
\titlespacing\section{0pt}{12pt plus 4pt minus 2pt}{0pt plus 2pt minus 2pt}
\titlespacing\subsection{0pt}{12pt plus 4pt minus 2pt}{0pt plus 2pt minus 2pt}
\titlespacing\subsubsection{0pt}{12pt plus 4pt minus 2pt}{0pt plus 2pt minus 2pt}

\usepackage{amsmath}
\usepackage{amssymb}
\usepackage{graphicx}
\usepackage{verbatim}    % For comment
\usepackage[shortlabels]{enumitem}
\usepackage[paper=a4paper, marginparwidth=0 cm, marginparsep=0 cm, top=2.5 cm, bottom=2.5 cm, left=3 cm, right=3 cm, includemp]{geometry}
\usepackage[pdftex, pdfstartview={FitH}, pdfnewwindow=true, colorlinks=true, citecolor=blue, filecolor=blue, linkcolor=blue, urlcolor=blue, pdfpagemode=UseNone]{hyperref}

% Put module code and last-modified date in footer
\usepackage{fancyhdr}
\pagestyle{fancy}
\fancyhf{}
\renewcommand{\headrulewidth}{0pt}
\cfoot{{\small \thisweek}\hfill \thepage\hfill {\small \moddate}}

% Hopefully address Canvas complaints about pdf tagging
%\usepackage[tagged]{accessibility}
% ------------------------------------------------------------------



% ------------------------------------------------------------------
% Shortcuts
\newcommand{\be}{\ensuremath{\beta} }
\newcommand{\De}{\ensuremath{\Delta} }
\newcommand{\Om}{\ensuremath{\Omega} }
\newcommand{\vev}[1]{\ensuremath{\left\langle #1 \right\rangle} }
\newcommand{\pderiv}[2]{\ensuremath{\frac{\partial #1}{\partial #2}} }
% ------------------------------------------------------------------



% ------------------------------------------------------------------
\begin{document}
\newcommand{\thisweek}{MATH327 Extra (Fugacity)}
\newcommand{\moddate}{Last modified 21 May 2021}
\begin{center}
  {\Large \textbf{MATH327: Statistical Physics, Spring 2021}} \\[12 pt]
  {\Large \textbf{Extra practice \ --- \ Fugacity and particle number}} \\[24 pt]
\end{center}

Consider a system governed by the grand-canonical ensemble with temperature $T = 1/ \be$ and chemical potential $\mu$.
Let $Z_g(T, \mu)$ denote the system's grand-canonical partition function, while $Z_N(T)$ indicates the partition function for a canonical ensemble with $N$ particles.
When these particles are classical and indistinguishable, we have seen
\begin{equation*}
  Z_N = \frac{1}{N!} Z_1^N,
\end{equation*}
where $Z_1(T)$ is the canonical partition function for a single particle.

\begin{enumerate}[label={(\alph*)}]
  \item Defining the fugacity $\xi \equiv e^{\be \mu} = e^{\mu / T}$, show
        \begin{equation*}
          Z_g(T, \mu) = \sum_{N = 0}^{\infty} \xi^N \, Z_N(T).
        \end{equation*}

  \item Derive a relation between the average particle number $\vev{N}$ and
        \begin{equation*}
          \xi \pderiv{}{\xi} \Om,
        \end{equation*}
        where $\Om = -T \log Z_g$ is the grand-canonical potential.
        It might be convenient to change variables to $x \equiv \log \xi = \be \mu$ with $\displaystyle \pderiv{}{x} = \xi\pderiv{}{\xi}$.

  \item Similarly, derive a relation between the squared particle number fluctuations
        \begin{equation*}
          \vev{\De N^2} \equiv \vev{\left(N - \vev{N}\right)^2}
        \end{equation*}
        and the second derivative
        \begin{equation*}
          \left(\xi \pderiv{}{\xi}\right)^2 \Om.
        \end{equation*}

  \item Specializing to indistinguishable classical particles, show
        \begin{equation*}
          Z_g(T, \mu) = \exp\left[\xi Z_1(T)\right].
        \end{equation*}

  \item Use the previous parts to derive the following result for the relative particle number fluctuations of indistinguishable classical particles:
        \begin{equation*}
          \frac{\sqrt{\vev{\De N^2}}}{\vev{N}} = \frac{1}{\sqrt{\vev{N}}}.
        \end{equation*}
        As discussed in the solutions for the second homework assignment, this relation connects the canonical and grand-canonical ensembles in the large-$N$ thermodynamic limit.
\end{enumerate}

\end{document}
% ------------------------------------------------------------------
