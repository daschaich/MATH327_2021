% ------------------------------------------------------------------
\documentclass[12 pt]{article} % A4 paper set by geometry package below
\pagenumbering{arabic}
\setlength{\parindent}{10 mm}
\setlength{\parskip}{12 pt}

% Nimbus Sans font should be reasonably legible
\usepackage{helvet}
\renewcommand{\familydefault}{\sfdefault}
\usepackage[T1]{fontenc}  % Without this \textsterling produces $

% Section header spacing
\usepackage{titlesec}
\titlespacing\section{0pt}{12pt plus 4pt minus 2pt}{0pt plus 2pt minus 2pt}
\titlespacing\subsection{0pt}{12pt plus 4pt minus 2pt}{0pt plus 2pt minus 2pt}
\titlespacing\subsubsection{0pt}{12pt plus 4pt minus 2pt}{0pt plus 2pt minus 2pt}

\usepackage{amsmath}
\usepackage{amssymb}
\usepackage{graphicx}
\usepackage{verbatim}    % For comment
\usepackage[paper=a4paper, marginparwidth=0 cm, marginparsep=0 cm, top=2.5 cm, bottom=2.5 cm, left=3 cm, right=3 cm, includemp]{geometry}
\usepackage[pdftex, pdfstartview={FitH}, pdfnewwindow=true, colorlinks=true, citecolor=blue, filecolor=blue, linkcolor=blue, urlcolor=blue, pdfpagemode=UseNone]{hyperref}

%\usepackage{amsfonts}
%\usepackage{float}
\usepackage{framed,color}
\usepackage{fancybox}
\usepackage{varwidth}
\definecolor{shadecolor}{rgb}{1,0.8,0.3}
\usepackage[framemethod=tikz]{mdframed}

% Put module code and last-modified date in footer
\usepackage{fancyhdr}
\pagestyle{fancy}
\fancyhf{}
\renewcommand{\headrulewidth}{0pt}
\cfoot{{\small \thisweek}\hfill \thepage\hfill {\small \moddate}}
% ------------------------------------------------------------------



% ------------------------------------------------------------------
% Shortcuts
\newcommand{\cE}{\ensuremath{\mathcal E} }
\newcommand{\cF}{\ensuremath{\mathcal F} }
\newcommand{\Nbb}{\ensuremath{\mathbb N} }
\newcommand{\cO}{\ensuremath{\mathcal O} }
\newcommand{\Rbb}{\ensuremath{\mathbb R} }
\newcommand{\Xbar}{\ensuremath{\overline X} }
\newcommand{\al}{\ensuremath{\alpha} }
\newcommand{\be}{\ensuremath{\beta} }
\newcommand{\De}{\ensuremath{\Delta} }
\newcommand{\de}{\ensuremath{\delta} }
\newcommand{\si}{\ensuremath{\sigma} }
\newcommand{\om}{\ensuremath{\omega} }
\newcommand{\Om}{\ensuremath{\Omega} }
\newcommand{\lra}{\ensuremath{\longrightarrow} }
\newcommand{\Lra}{\ensuremath{\Longrightarrow} }
\newcommand{\llra}{\ensuremath{\longleftrightarrow} }
\newcommand{\X}{\ensuremath{\!\times\!} }
\newcommand{\vev}[1]{\ensuremath{\left\langle #1 \right\rangle} }
\newcommand{\eq}[1]{Eq.~\ref{#1}}
\newcommand{\fig}[1]{Figure~\ref{#1}}
\newcommand{\secref}[1]{Section~\ref{#1}}
\newcommand{\TODO}[1]{\textcolor{red}{\textbf{#1}}}
% ------------------------------------------------------------------



% ------------------------------------------------------------------
\begin{document}
\newcommand{\thisweek}{MATH327 information}
\newcommand{\moddate}{Last modified 15 Jan.~2021}
\thispagestyle{empty}
\begin{center}
  {\LARGE \textbf{MATH327: Statistical Physics}} \\[6 pt]
  \textbf{David Schaich \qquad\qquad\qquad\qquad Spring 2021} \\[48 pt]
  {\LARGE LECTURE \ NOTES} \\[6 pt]
  \moddate
\end{center}
\renewcommand{\contentsname}{}
\setcounter{tocdepth}{1}
\tableofcontents
% ------------------------------------------------------------------



% ------------------------------------------------------------------
\newpage
\setcounter{section}{0}
\addcontentsline{toc}{section}{Module information}
\section*{Module information}
\subsection*{Coordinator}
\begin{description}
  \setlength{\itemsep}{1pt}
  \setlength{\parskip}{0pt}
  \setlength{\parsep}{0pt}
  \item[\qquad] Dr David Schaich
  \item[\qquad] Mathematics Building Room 124 (Theoretical Physics Wing)
  \item[\qquad] \href{mailto:david.schaich@liverpool.ac.uk}{david.schaich@liverpool.ac.uk}
  \item[\qquad] \href{http://www.davidschaich.net}{www.davidschaich.net} \\
  \item[\qquad] Office hours: \TODO{TBD} and by request via \href{https://calendly.com/daschaich}{calendly.com/daschaich}
\end{description}
% ------------------------------------------------------------------



% ------------------------------------------------------------------
\subsection*{Logistics, lecture notes, and learning strategy}
All resources for this module will be gathered at its Canvas site.
\TODO{...notes-driven delivery with video recordings for each week providing introductory motivation/overview and concluding summary of main learning objectives...}
\TODO{...notes have gaps --- attempt to fill these in while working through --- will be discussed during synchronous sessions, and any that prove particularly tricky I may address through supplemental follow-up video recordings...}
%In addition to these lecture notes, these resources include explanatory videos, sample problems \TODO{and more...}
\TODO{...such supplements can also be requested outside of synchronous sessions...} % TODO: Need to be provided a week in advance?...
%\TODO{...can supplement with recorded lectures from last year on stream.liv.ac.uk...}

\TODO{...no need to memorize equations---numbers mostly used in order to refer back to past work in later derivations... colored boxes highlight main learning objectives to be understood...}

\TODO{...module organized around statistical ensembles...}
\TODO{...mostly focusing on systems whose constituents (which may be particles, balls, spins or other objects) do not interact with each other...}

\TODO{...same content as last year to make sure expectations are reasonable...} % TODO: Last year spent six timetabled hours on computer project --- add that to schedule or leave it separate???...  Could merge weeks 8--9 and move the grand-canonical ensemble after spring break, but I'm worried students may revolt if a week is fully given over to code training...
\TODO{...load may fluctuate from week to week...}

\subsubsection*{Expected background}
\TODO{Quantum and combinatorics not assumed... do expect binomial coefficients and gaussian integrals...}

\subsubsection*{Version control and co-creation}
The \LaTeX\ source for these lecture notes is kept under version control at \\
\centerline{\href{https://github.com/daschaich/MATH327_2021}{github.com/daschaich/MATH327\_2021}}
If any changes need to be made to these notes during the term (to correct mistakes or add supplemental information), the GitHub interface provides an easy way to see what changed and when.
In addition, you are also welcome to use GitHub to report issues and create pull requests to address them.
Such co-creation is entirely optional.
If you are interested in learning more about version control with \texttt{git}, the \href{https://software-carpentry.org}{Software Carpentry} project provides free resources at \\
\centerline{\href{https://swcarpentry.github.io/git-novice/}{swcarpentry.github.io/git-novice/}}
% ------------------------------------------------------------------



% ------------------------------------------------------------------
\subsection*{Assessment and academic integrity}
The assessment workload has also been kept the same as last year, but in light of this year's hybrid teaching and learning approach the weighting of assignments has been spread more evenly across the term to avoid a high-stakes final assessment.
The following deadlines for the in-term assignments have been coordinated within the Department to minimize pile-up: \\[-24 pt]
\begin{description}
  \item[15\%] A homework assignment covering weeks 1--3, due \TODO{Friday, 5 March (TBC)}
  \item[20\%] A computer-based project divided into two equally weighted parts, the first due \TODO{Friday, 19 March (TBC)} and the second due \TODO{Friday, 16 April (TBC)}
  \item[15\%] A homework assignment covering weeks 4--8, due \TODO{Friday, 30 April (TBC)}
  \item[50\%] A final assessment to be centrally scheduled within the period 17 May through 4 June
\end{description}

Because the computer-based project will be done remotely, rather than in a computer lab on campus, you are free to complete it using the programming language of your choice.
Training in the necessary programming concepts will be provided using \href{https://www.python.org}{Python}, the free programming language recommended for the project.
If you have difficulty setting up Python on your device, you can run it for free online at \href{https://repl.it/languages/python3}{repl.it}.
(Make sure to keep a copy of the code to submit for assessment!)
Alternative languages could include \href{https://en.wikipedia.org/wiki/C_(programming_language)}{C}, \href{https://www.r-project.org}{R}, or even \href{https://matlab.mathworks.com}{MATLAB} (through the University's site license).
Maple may struggle to handle parts of the project.

I encourage you to discuss the in-term assignments with each other, since discussing and debating concepts and procedures is a very effective way to learn.
The examination must be done on your own, and your submissions for all assignments must be your own work representing your own understanding.
Clear and neat presentations of your workings and the logic behind them will contribute to your mark.
It is unacceptable to copy solutions in part or in whole from other students (current or prior) or from other sources (commercial or otherwise).
Should you make use of resources beyond the module materials, these must be explicitly referenced in your work.

By now you should have successfully passed the Academic Integrity Tutorial and Quiz to affirm that you have read and understood the Academic Integrity Policy detailed in Appendix L of the Code of Practice on Assessment.
If you have any questions about what is or is not acceptable under this policy, please ask me or Departmental Assessment Officer Kamila Zychaluk.
In all cases, the spirit of the Academic Integrity Policy should take precedence over legalistic convolutions of the text.
% ------------------------------------------------------------------



% ------------------------------------------------------------------
\subsection*{Additional resources}
The lists of further reading below provide some optional additional resources that may be helpful.
Hyperlinks lead either to the resource itself or to the corresponding record page in our library.

\noindent\textbf{These books and lecture notes are roughly at the level of this module:} \\[-24 pt]
\begin{enumerate}
  \item David Tong, \href{https://www.damtp.cam.ac.uk/user/tong/statphys.html}{\textit{Lectures on Statistical Physics}} (2012), \\ www.damtp.cam.ac.uk/user/tong/statphys.html
  \item Daniel V.~Schroeder, \textit{An Introduction to Thermal Physics} (first edition, 2000)
  \item C.~ Kittel and H.~Kroemer, \textit{Thermal Physics} (second edition, 1980)
  \item F.~Reif, \textit{Fundamentals of Statistical and Thermal Physics} (first edition, 1965)
\end{enumerate}

\noindent\textbf{These books are more advanced and more specialized, but can be useful to consult concerning specific questions or topics:} \\[-24 pt]
\begin{enumerate}
  \setcounter{enumi}{4}
  \item Weinan E, Tiejun Li and Eric Vanden-Eijnden, \textit{Applied Stochastic Analysis} (first edition, 2019)
  \item R.~K.~Pathria, \textit{Statistical Mechanics} (second edition, 1996)
  \item Sidney Redner, \textit{A Guide to First-Passage Processes} (first edition, 2007)
  \item Pavel L.~Krapivsky, Sidney Redner and Eli Ben-Naim, \textit{A Kinetic View of Statistical Physics} (first edition, 2010)
  \item Kerson Huang, \textit{Statistical Mechanics} (second edition, 1987)
  \item Michael Plischke and Birger Bergersen, \textit{Equilibrium Statistical Physics} (third edition, 2005)
  \item L.~D.~Landau and E.~M.~Lifshitz, \textit{Statistical Physics, Part 1} (third edition, 1980)
\end{enumerate}

\noindent\textbf{Maple and MATLAB resources:} \\[-24 pt]
\begin{enumerate}
  \setcounter{enumi}{11}
  \item Ian Thompson, \href{https://library.liv.ac.uk/record=b4395758~S8}{\textit{Understanding Maple}} (ebook edition, 2017) \\
        Related videos: \href{https://stream.liv.ac.uk/4k67bdzt}{\textit{Running Maple for the first time}} (stream.liv.ac.uk/4k67bdzt) \\
        \textcolor{white}{Related videos:} \href{https://stream.liv.ac.uk/7pjge23a}{\textit{Configuring Maple}} (stream.liv.ac.uk/7pjge23a)
  \item Stormy Attaway, \textit{MATLAB: A Practical Introduction to Programming and Problem Solving} (third edition, 2013)
  \item B.~Barnes and G.~R.~Fulford, \textit{Mathematical Modelling with Case Studies: Using Maple and MATLAB} (third edition, 2014)
\end{enumerate}

Finally, there is a vast constellation of purely online resources, such as \href{https://en.wikipedia.org/wiki/Statistical_physics}{Wikipedia}.
These are often fine places to \emph{start} learning about a subject, but may be terrible places to \emph{stop}.
% ------------------------------------------------------------------



% ------------------------------------------------------------------
\newpage
% ------------------------------------------------------------------
\renewcommand{\thisweek}{MATH327 Week 1}
\renewcommand{\moddate}{Last modified 16 Jan.~2021}
\setcounter{section}{1}
\section*{Week 1: Central limit theorem and diffusion}
\addcontentsline{toc}{section}{Week 1: Central limit theorem and diffusion}

\subsection*{Introductory remarks: What is Statistical Physics?}
Mathematical sciences such as physics aim to determine the laws of nature and understand how these govern experimental observations---both in everyday circumstances and under extreme conditions.
This mathematical understanding is typically guided by reproducing a set of observations, with the resulting framework then used to make predictions for other ``observables''.

Over the past few centuries this process has been tremendously successful, with theoretical physics accurately predicting experimental and observational results from sub-atomic through to extra-galactic scales.
Modern physics labs can create a vacuum better than in outer space and the coldest temperatures in the known universe, as well as going to the other extreme to reach temperatures of millions of degrees and pressures millions of times atmospheric pressure at sea level.
Amazingly, many aspects of these realms of physics can be theoretically described by mathematics developed centuries ago.\footnote{Eugene Wigner's famous article, ``\href{https://en.wikipedia.org/wiki/The_Unreasonable_Effectiveness_of_Mathematics_in_the_Natural_Sciences}{The Unreasonable Effectiveness of Mathematics in the Natural Sciences}'' (1960), and subsequent work in the philosophy of physics, elaborates on why this may be considered `amazing'.  These lecture notes will not comment extensively on philosophy.}

The domain of \textbf{statistical physics} is one in which simple mathematical principles enable amazing predictive capabilities.
Initially developed in the nineteenth century, statistical physics remains a core component of modern physics, and will retain this position in years to come.
The foundations of statistical physics lie in the use of probability theory to mathematically describe experimental observations and corresponding laws of nature that involve stochastic randomness rather than being perfectly predictable.

The lack of perfect predictability in statistical physics is a matter of practicality rather than one of principle.
It results from working with a large number of degrees of freedom (i.e., a large number of independent objects such as particles or balls).
For example, Avogadro's number $N_A \approx 6.022\times 10^{23}$ is the large number of molecules in everyday amounts of familiar substances---about 18~grams of water or about 22~litres of air at sea-level atmospheric pressure ($\approx$$101$~kPa). % 22.4 litres for 1atm=101.325~kPa, 22.71 litres for 100~kPa at $O^{\circ}$~C
Specifying the positions and velocities of $\sim$$10^{23}$ objects would require far more information than could be stored even in the memory of the biggest existing supercomputers.
Statistical physics instead produces simple mathematical descriptions of large-scale properties such as temperature, pressure and diffusion, which are generally of such outstanding quality that the underlying `randomness' is effectively undetectable.

Historically, the difficulty detecting the stochastic processes underlying such \textit{thermodynamic} properties made it challenging to convince skeptics that atoms and molecules really exist.
Ludwig Boltzmann, a prominent early developer of statistical physics, endured a constant struggle to defend his ideas, which likely contributed to his deteriorating mental health and eventual suicide in 1906.
A crucial advance to convincingly establish the existence of atoms was Albert Einstein's use of statistical physics to explain the observed ``\href{https://en.wikipedia.org/wiki/Brownian_motion}{Brownian motion}'' of particles suspended in fluids---this work was part of Einstein's ``miracle year'' in 1905, along with special relativity and early contributions to quantum physics.
More modern applications of statistical physics include explaining why stars don't collapse under the `weight' of their own gravity, and identifying effects of dark matter in temperature fluctuations observable in the \textit{cosmic microwave background} lingering from the early years of the universe.

For this week we will focus on some of the foundational mathematics that will underlie our later development and application of statistical ensembles.
Looking back to Boltzmann's times, we can consider the following question one of his opponents might have asked:
\textit{If the pressure of a gas in a container results from molecules stochastically colliding with the walls of that container and pushing them out, then how can the pressure be so stable and reproducible, rather than itself fluctuating stochastically?}
The mathematical answer lies in the \textbf{law of large numbers} and the \textbf{central limit theorem}, which this week we will learn and apply to the physics of diffusion in one dimension.
% ------------------------------------------------------------------



% ------------------------------------------------------------------
\subsection{\label{sec:prob}Probability foundations}
Let's begin by placing some familiar everyday concepts into a more formal mathematical framework through the following definitions: \\[-24 pt]
\begin{itemize}
  \item A (random) \textbf{experiment} \cE involves setting up, manipulating and/or observing some (physical or hypothetical) system with some element of randomness.
        Flipping a coin is a simple random experiment.
        In the context of the statistical ensembles that will be the focus of this module, the typical experiment will be simply allowing a collection of particles to evolve in time, subject to certain constraints.
  \item Each time an experiment is performed, the world comes out in some \textbf{state} $\om$.
        The definition of the experiment and the state must include all objects of interest, and may include more besides.
        When flipping a coin, for example, the full state could contain information not only about the final orientation of the coin, but also about its position---did it land on the floor or on a cat?
  \item The \textbf{set of all states} \Om collects all possible states \om that the given experiment \cE can produce, and is therefore intricately tied to \cE itself.
  \item We are generally not interested in all aspects of the full state $\om$.
        For example, we won't care where a flipped coin lands.
        Instead we're typically only interested in whether it lands heads up or tails up---and we may want to set aside any state that doesn't cleanly map on to those options.
        The \textbf{measurement} $X(\om)$ extracts and quantifies this information, acting as a function that maps the state \om to a number that we can mathematically manipulate.
        If we repeat a fixed experiment \cE many times and carry out the measurement $X$ on each resulting state $\om$, we will obtain a sequence of numbers $X(\om)$ that behave as a \textit{random variable}.
  \item Acting with the measurement $X$ on all of the possible states in the set $\Om$ defines the \textbf{set of all outcomes} (or \textbf{outcome space}) $A$:
        \begin{equation*}
          X: \Om \to A.
        \end{equation*}
        That is, $A$ collects all possible measurement results that the given experiment \cE and measurement $X$ can produce.
        $A$ can be finite, countably infinite, or uncountably infinite (i.e., continuous).
  \item Finally, defining an \textbf{event} to be any subset of the set of all outcomes $A$, we further group these subsets together to define a \textbf{set of events} $\cF$.
\end{itemize}

Let's consider some \textbf{examples} to clarify these definitions.
With an experiment of rolling a six-sided die and measuring the number ($1$--$6$) that comes out on top, what is the set of all outcomes $A$, and what additional information could be present in the set of all states $\Om$?
\begin{mdframed}
  \ \\[100 pt]
\end{mdframed}
What is the outcome space $A$ if we toss a coin four times and measure whether it lands heads up ($H$) or tails up ($T$)?
\begin{mdframed}
  \ \\[100 pt]
\end{mdframed}
\newpage % WARNING: FORMATTING BY HAND
\noindent What information would characterize a state \om for a gas of $6\X 10^{23}$ argon atoms in a container?
\begin{mdframed}
  \ \\[100 pt]
\end{mdframed}

We can generalize the concept of measurement by introducing a unique number as a \textit{label} to characterize each state \om in the set $\Om$.
This would provide a label function $L(\om)$ as a random variable.
Our condition of uniqueness makes $L(\om)$ isomorphic, so that the label can be used interchangeably with the full state,
\begin{equation*}
  \om \llra L(\om).
\end{equation*}
While the measurements $X(\om)$ we consider will generally not produce a unique number for each $\om$, we will design them (as best we can) precisely to remove irrelevant information that doesn't interest us.
Ignoring that irrelevant information leaves us free to interchange the set of outcomes $A$ for the set of states $\Om$, which we will do from now on.
(Some textbooks may never distinguish between $A$ vs \Om in the first place, though this can be a source of confusion.)

We are now prepared for the final foundational definition in this section, the \textbf{probability} $P$ that an event in the set \cF has of occurring.
Mathematically, $P$ is a \textit{measure function},
\begin{equation*}
  P: \cF \to [0, 1],
\end{equation*}
which must satisfy the following two requirements: \\[-24 pt]
\begin{enumerate}
  \item The probability of a countable union of mutually exclusive events must equal the countable sum of the probabilities of each of these events.
  \item The probability of the outcome space ($\cF = A$) must equal $1$ (even if $A$ is uncountable).
        This simply means that the experiment \cE must produce an outcome.
        If no outcome were produced, it would not make sense to say that the experiment had occurred.
\end{enumerate}
Combining the outcome space, event space and probability measure gives us a \textit{probability space} $(A, \cF, P)$.

For \textbf{example}, consider an experiment that can only produce $N$ possible states, so that
\begin{equation*}
  \Om = \left\{\om_1, \om_2, \cdots, \om_N\right\}.
\end{equation*}
If two states are identical, $\om_i = \om_j$, they must produce the same measurement outcomes $X(\om_i) = X(\om_j)$, which implies the contra-positive
\begin{equation*}
  X(\om_i) \ne X(\om_j) \qquad \Lra \qquad \om_i \ne \om_j.
\end{equation*}
On the other hand, as described above, it is possible to have $X(\om_i) = X(\om_j)$ even when $\om_i \ne \om_j$.
This means that the size $n$ of the outcome space $A$ may be smaller than the size of $\Om$, $n \leq N$.
We can write
\begin{equation*}
  A = \left\{X_1, X_2, \cdots, X_n\right\},
\end{equation*}
where each $X_{\al}$ is distinct and its index does \textit{not} necessarily correspond to that on $\om_i$.
We can take the individual $X_{\al}$ themselves to be the events we're interested in, with an overall event space
\begin{equation*}
  \cF = \left\{X_1, X_2, \cdots, X_n\right\} = A.
\end{equation*}
These events are all mutually exclusive by construction, so if we assign them probabilities
\begin{equation*}
  P(X_{\al}) \equiv p_{\al} \qquad \mbox{for } \al = 1, \cdots, n,
\end{equation*}
then the above requirements on probabilities demand that for any $\al \ne \be$ we have
\begin{align*}
  P(X_{\al} \mbox{ or } X_{\be}) & = p_{\al} + p_{\be} \\
  P(X_1 \hbox{ or } X_2 \hbox{ or } \cdots \mbox{ or } X_n) & = \sum_{\al = 1}^n p_{\al} = 1.
\end{align*}

Similarly choosing an event space $\cF = A$ for the six-sided die considered in an earlier gap, what are the probabilities $p_1$ through $p_6$ that result from assuming the die is \textit{fair} (meaning that each outcome is equally probable)?
\begin{mdframed}
  \ \\[100 pt]
\end{mdframed}
\newpage % WARNING: FORMATTING BY HAND
\noindent Again taking $\cF = A$ for the case of tossing a coin four times, what are the probabilities $p_{\al}$ that result from assuming the coin is fair?
If we instead consider the event space
\begin{equation*}
  \cF = \left\{\mbox{equal number of } H \mbox{ and } T, \mbox{ different numbers of } H \mbox{ and } T\right\},
\end{equation*}
what are the probabilities $p_{\text{equal}}$ and $p_{\text{diff}}$ for the two events in this $\cF$?
\begin{mdframed}
  \ \\[100 pt]
\end{mdframed}

\begin{minipage}{0.4\textwidth}
  \includegraphics[width=0.75\textwidth]{figs/roulette.pdf}
\end{minipage}%
\begin{minipage}{0.5\textwidth}
  The standard European roulette wheel shown to the left (\href{https://www.vecteezy.com/vector-art/658761-casino-roulette-wheel}{source}) has 37 pockets labelled ``0'' through ``36''.
  18 of these pockets are coloured red, 18 are coloured black and 1 (pocket ``0'') is coloured green.
\end{minipage}

\noindent What is the outcome space $A$ for a spin of the roulette wheel?
With $\cF = A$, what are the probabilities $p_{\al}$ for a fair wheel?
With
\begin{equation*}
  \cF = \left\{\mbox{ball in a red pocket, ball in a black pocket, ball in the green pocket}\right\},
\end{equation*}
what are the corresponding probabilities $p_{\text{red}}$, $p_{\text{black}}$ and $p_{\text{green}}$?
\begin{mdframed}
  \ \\[100 pt]
\end{mdframed}

We conclude this section with two \textbf{comments} on the process of assigning probabilities to events (which is called \textit{modelling}): \\[-24 pt]
\begin{itemize}
  \item We saw above that \textit{symmetries} are a powerful way to constrain probabilities.
        The symmetry between the six sides of a fair die, the two sides of a fair coin, and the 37 pockets of a fair roulette wheel each sufficed to completely fix the corresponding probabilities $p_{\al}$.
  \item Modelling can also be guided by empirical information obtained by repeating an experiment many times.
        For example, if we don't know whether a set of dice are fair, we will be able to infer their probabilities $p_{\al}$ (with a certain confidence level) by rolling them enough times.
        The need to repeat the experiment many times comes from the law of large numbers, to which we now turn.
\end{itemize}
% ------------------------------------------------------------------



% ------------------------------------------------------------------
\subsection{\label{sec:LLN}Law of large numbers}
Let's continue considering the setup introduced above, with
\begin{equation}
  \label{eq:finite_set}
  \cF = A = \left\{X_1, X_2, \cdots, X_n\right\}
\end{equation}
for finite $n$, and probabilities $p_{\al} = P(X_{\al})$ that obey
\begin{align*}
  p_{\al} & \in [0, 1] &
  \sum_{\al = 1}^n p_{\al} & = 1.
\end{align*}
We can generalize this notation by writing instead
\begin{equation*}
  \sum_{X \in A} P(X) = 1,
\end{equation*}
which provides simple expressions for the \textbf{mean} $\mu$ and \textbf{variance} $\si^2$ of the probability space,
\begin{align}
              \mu = \vev{X} = & \sum_{X \in A} X P(X)            \label{eq:mean} \\
  \si^2 = \vev{(X - \mu)^2} = & \sum_{X \in A} (X - \mu)^2 P(X). \label{eq:var}
\end{align}
The angle bracket notation indicates the \textbf{expected} (or \textbf{expectation}) \textbf{value} with general definition
\begin{equation}
  \label{eq:expect_disc}
  \vev{f(X)} = \sum_{X \in A} f(X) P(X),
\end{equation}
which is a linear operation,
\begin{equation*}
  \vev{c\cdot f(X) + g(X)} = c\vev{f(X)} + \vev{g(X)}.
\end{equation*}
\newpage % WARNING: FORMATTING BY HAND
\noindent The square root of the variance, $\sqrt{\si^2} = \si$, is the \textbf{standard deviation}.
What is \si expressed in terms of $\vev{X^2}$ and $\vev{X}^2$?
\begin{mdframed}
  \ \\[100 pt]
\end{mdframed}

We now define a new experiment that consists of \textit{repeating} the original experiment $R$ times, with each repetition independent of all the others.
Using the same measurement as before for each repetition, we obtain a new outcome space that we can call $B$.
For $R = 4$, what are some representative outcomes in the set $B$?
What is the total size of $B$?
\begin{mdframed}
  \ \\[100 pt]
\end{mdframed}

Each outcome in $B$ contains $R$ different $X^{(r)} \in A$ (with $\vev{X^{(r)}} = \mu$ and $\vev{(X^{(r)} - \mu)^2} = \si^2$), one for each repetition $r = 1, \cdots, R$.
Considering the case $R = 4$ for simplicity, any element of $B$ can be written as $X_i^{(1)} X_j^{(2)} X_k^{(3)} X_l^{(4)} \in B$ with corresponding probability
\begin{equation*}
  P_B\left(X_i^{(1)} X_j^{(2)} X_k^{(3)} X_l^{(4)}\right) = P_A\left(X_i^{(1)}\right) P_A\left(X_j^{(2)}\right) P_A\left(X_k^{(3)}\right) P_A\left(X_l^{(4)}\right),
\end{equation*}
using subscripts to distinguish between the single-experiment ($A$) and repeated-experiment ($B$) probability spaces.

Averaging over all $R$ repetitions defines the \textit{arithmetic mean}
\begin{equation}
  \label{eq:ave}
  \Xbar_R = \frac{1}{R} \sum_{r = 1}^R X^{(r)}.
\end{equation}
Unlike the true mean $\mu$, the arithmetic mean $\Xbar_R$ is a random variable---a number that may be different for each element of $B$.
That said, $\Xbar_R$ and $\mu$ are certainly related, and so long as the standard deviation exists (i.e., $\si^2$ is finite), this relation can be proved rigorously in the limit $R \to \infty$.\footnote{In the computer-based project we will numerically investigate a case with divergent $\si^2$.}

Here we will not be fully rigorous, and take it for granted that
\begin{equation*}
  \vev{\left(X^{(i)} - \mu\right)\left(X^{(j)} - \mu\right)} = \si^2 \de_{ij} = \left\{\begin{array}{ll}\si^2 & \mbox{for } i = j \\ 0 & \mbox{for } i \ne j\end{array}\right.,
\end{equation*}
where the \textit{Kronecker delta} $\de_{ij} = 1$ for $i = j$ and vanishes for $i \ne j$.
This is a consequence of the assumed independence of the different repetitions.
Using this result and the relation $\big(\sum_i a_i\big)\big(\sum_j b_j\big) = \sum_{i, j} \left(a_i b_j\right)$, express the following quantity in terms of \si and $R$:
\begin{mdframed}
  $\displaystyle \vev{\left(\frac{1}{R} \sum_{r = 1}^R X^{(r)} - \mu\right)^2} = $ \\[100 pt]
\end{mdframed}
You should find that your result vanishes in the limit $R \to \infty$, so long as $\si^2$ is finite.
Since the square makes this expectation value a sum of non-negative terms, it can vanish only if every one of those terms is individually zero.

\begin{shaded}
  This establishes the \textbf{law of large numbers}:
  \begin{equation}
    \lim_{R \to \infty} \frac{1}{R} \sum_{r = 1}^R X^{(r)} = \mu,
  \end{equation}
  where we have assumed $\vev{X^{(r)}} = \mu$ and $\vev{(X^{(r)} - \mu)^2} = \si^2$ are finite.
\end{shaded}
% ------------------------------------------------------------------



% ------------------------------------------------------------------
\subsection{\label{sec:probdist}Probability distributions}
It is not necessary to make the assumption (\eq{eq:finite_set}) that our outcome space contains only a countable number of possible outcomes.
The considerations above continue to hold even if the random variable $X$ is a continuous real number.
In this case, however, the identification of probabilities with outcomes is slightly more complicated, which will be relevant when we consider the central limit theorem in the next section.

When the outcome can be any number on the real line, the fundamental object is a \textbf{probability distribution} (or \textbf{density function}) $p(x)$ defined for all $x \in \Rbb$.
Starting from this density, a probability is determined by integrating over a given interval.
Calling this interval $[a, b]$, the integration produces the probability that the outcome $X$ lies within the interval,
\begin{equation*}
  P\left(a \leq X \leq b\right) = \int_a^b p(x) \; dx.
\end{equation*}

We similarly generalize the definition of an expectation value (\eq{eq:expect_disc}) to an integral over the entire domain of the  probability distribution,
\begin{equation*}
  \vev{f(x)} = \int f(x) \; p(x) \; dx.
\end{equation*}
We will omit the limits on integrals over the entire domain, so for $x \in \Rbb$ we implicitly have $\int dx = \int_{-\infty}^{\infty} dx$.
An important set of expectation values is
\begin{equation}
  \label{eq:expect_cont}
  \vev{x^{\ell}} = \int x^{\ell} \; p(x) \; dx,
\end{equation}
which provides the mean and variance of the probability distribution $p(x)$, through generalizations of Eqs.~\ref{eq:mean}--\ref{eq:var}:
\begin{align}
  \label{eq:mean_var}
  \mu   & = \vev{x} = \int x \; p(x) \; dx &
  \si^2 & = \vev{x^2} - \vev{x}^2.
\end{align}
The expression for the variance should be familiar from your determination of the standard deviation in an earlier gap.
Unless stated otherwise, we will assume the mean and variance are both finite for the probability distributions we consider.
% ------------------------------------------------------------------



% ------------------------------------------------------------------
\subsection{Central limit theorem}
The central limit theorem is a major result of probability theory.
Over the years it has been expressed in several equivalent ways, and there are also many distinct variants of the theorem accommodating different conditions and assumptions.
In this module we are interested in applying rather than proving the central limit theorem; the curious can find proofs in many textbooks.

The version of the theorem we use in this module assumes we have $N$ independent random variables $x_1, \cdots, x_N$, each of which has the same (finite) mean $\mu$ and variance $\si^2$.
(Such random variables are said to be \textit{identically distributed}, and a common way to obtain them is to repeat an experiment $N$ times, as we considered in \secref{sec:LLN}.)
Just as in \eq{eq:ave}, the sum
\begin{equation*}
  s = \sum_{i = 1}^N x_i
\end{equation*}
is itself a random variable.

\begin{shaded}
  The \textbf{central limit theorem} states that for large $N \gg 1$ the probability distribution for $s$ is
  \begin{equation}
    p(s) \approx \frac{1}{\sqrt{2\pi N\si^2}} \exp\left[-\frac{(s - N\mu)^2}{2N\si^2}\right],
  \end{equation}
  with the approximation becoming exact in the $N \to \infty$ limit.
\end{shaded}
In addition to asserting that the collective behaviour of many independent and identically distributed random variables $x_i$ is governed by a \textbf{normal} (or \textbf{gaussian}) \textbf{distribution}, the central limit theorem further specifies the precise form of this distribution in terms of the mean and variance of \textit{each individual} $x_i$.

As an \textbf{example} to illustrate the applicability of the central limit theorem even for a modest $N = 5$, consider the roulette wheel discussed in \secref{sec:prob}.
A simple game of roulette would let us place bets on whether or not the ball will end up in a red- or black-coloured pocket: If we bet correctly we get back twice the money we put in; otherwise we lose our money.
Define our (potentially negative) \textit{gain} to be the amount we receive minus the amount we spend on bets.

Suppose we place \textsterling5 bets on `black' for each of $N$ spins of the roulette wheel.
What are the probabilities and gains of winning and of losing for any single one of those spins?
Letting $W = 0, \cdots, N$ be the number of spins where we win, show that our total gain is $G_W = 10W - 5N$.
\begin{mdframed}
  \ \\[100 pt]
\end{mdframed}
Recall that the number of different ways we could win $W$ times out of $N$ attempts is given by the binomial coefficient
\begin{equation*}
  \left(\begin{array}{c} N \\ W \end{array}\right) = \frac{N!}{W! \; (N - W)!},
\end{equation*}
with $0! = 1$.
Setting $N = 5$, what are the six probabilities $p_0$--$p_5$ that we win $W = 0, \cdots, 5$ times?
What is the general expression for $p_i$?
\begin{mdframed}
  \ \\[100 pt]
\end{mdframed}

\newpage % WARNING: FORMATTING BY HAND
Now let's apply the central limit theorem to this setup.
What are the mean gain and its variance for a single spin of the wheel?
What is the resulting probability distribution $p(G)$ given by the central limit theorem for the gain after $N$ spins?
\begin{mdframed}
  \ \\[100 pt]
\end{mdframed}
In order to compare this approximation with the exact $p_i$ computed above, we need to extract probabilities by integrating over appropriate intervals as discussed in \secref{sec:probdist}.
\TODO{...}
% ------------------------------------------------------------------



% ------------------------------------------------------------------
\subsection{\label{sec:diffusion}Diffusion on a line}
As a more generic application of the central limit theorem, let's consider the behaviour of an object moving randomly in one dimension---say to the left or right on a line.
\TODO{...}
Such \textbf{random walks} appear frequently in mathematical modelling of stochastic phenomena (including Brownian motion).
\TODO{...}
% ------------------------------------------------------------------


\newpage
% ------------------------------------------------------------------
\renewcommand{\thisweek}{MATH327 Week 2}
\renewcommand{\moddate}{Last modified 2 Feb.~2021}
\setcounter{section}{2}
\setcounter{subsection}{0}
\phantomsection
\addcontentsline{toc}{section}{Week 2: Micro-canonical ensemble}
\section*{Week 2: Micro-canonical ensemble}

\subsection{First law of thermodynamics}
We begin this week by developing the concept of \textit{statistical ensembles} (introduced by \href{https://en.wikipedia.org/wiki/Josiah_Willard_Gibbs}{J.\ Willard Gibbs} in the early 1900s), building on the probability foundations we laid last week.
As forecast last week, we will be interested in `experiments' that simply allow a collection of degrees of freedom to evolve in time, subject to certain constraints.
At a given time $t_1$, the disposition of these degrees of freedom defines the state $\om_1$ of the system.
To consider a couple of examples, what would be a representative state for a system of $8$ \textit{spins} (arrows that can point either up or down) evenly spaced along a line?
What information would characterize the state of $N$ hydrogen (H$_2$) molecules in a container?
\begin{mdframed}
  \ \\[100 pt]
\end{mdframed}

At a different time $t_2$, the system's state $\om_2$ is most likely different from $\om_1$.
However, there are some \textit{measurements} we can perform on these states that do not change as the system evolves in time.
These measurements define \textit{conserved quantities}, an important example of which is the 

\TODO{Being written...}
% ------------------------------------------------------------------



% ------------------------------------------------------------------
\newpage % TODO: Placeholder...
\subsection{Entropy and its properties}
\TODO{Being written...}
% ------------------------------------------------------------------



% ------------------------------------------------------------------
\newpage % TODO: Placeholder...
\subsection{Temperature}
\TODO{Being written...}
% ------------------------------------------------------------------



% ------------------------------------------------------------------
\newpage % TODO: Placeholder...
\subsection{Heat exchange}
\TODO{Being written...}
% ------------------------------------------------------------------


\newpage
% ------------------------------------------------------------------
\renewcommand{\thisweek}{MATH327 Week 3}
\renewcommand{\moddate}{Last modified 15 Feb.~2021}
\setcounter{section}{3}
\setcounter{subsection}{0}
\phantomsection
\addcontentsline{toc}{section}{Week 3: Canonical ensemble}
\section*{Week 3: Canonical ensemble}

\subsection{The thermal reservoir}
\subsubsection{Replicas and occupation numbers}
While it is relatively easy to prevent particle exchange, for example by sealing gases inside airtight containers, it is not practical to forbid energy exchange as would be needed to fully isolate statistical systems.
Any thermal insulator is imperfect, and even in the deepest reaches of space we would still be bombarded by cosmic microwave radiation.
In practice it is more convenient to work with physical systems that are characterized by their (intensive) temperatures rather than their (extensive) internal energies.

\begin{shaded}
  This leads us to define a \textbf{canonical ensemble} to be a statistical ensemble characterized by its fixed temperature $T$ and conserved particle number $N$, with the temperature held fixed through contact with a \textbf{thermal reservoir}.
\end{shaded}

The second part of this definition connects the fixed temperature to the fundamental fact of energy conservation (the first law of thermodynamics).
This is done by proposing that our system of interest \Om is in thermal contact with a much larger external system $\Om_{\text{res}}$---the thermal reservoir, sometimes called a ``heat bath''.
The overall combined system $\Om_{\text{tot}} = \Om_{\text{res}} \otimes \Om$ is governed by the micro-canonical ensemble, with conserved total energy $E_{\text{tot}} = E_{\text{res}} + E \approx E_{\text{res}}$, while the energy $E$ of \Om is allowed to fluctuate.
The key qualitative idea is that, in thermodynamic equilibrium, \Om has a negligible effect on the overall system.
In particular, the temperature of that overall system---and therefore the temperature of $\Om$, by intensivity---is set by the reservoir and remains fixed even as $E$ fluctuates.
This effectively generalizes the setup we used to analyze heat exchange last week, where we saw that thermal contact causes a net flow of energy from hotter systems to colder systems, cooling the former by heating the latter.

The mathematical realization of this argument, as developed by Gibbs, proceeds by considering a well-motivated ansatz for the form of the thermal reservoir $\Om_{\text{res}}$.
\textcolor{green}{The goal}, which will be useful to keep in mind as we go through the lengthy analysis, is to show that the specific form of $\Om_{\text{res}}$ is ultimately irrelevant.
This will allow us to work directly with the system of interest, $\Om$, independent of the details of the thermal reservoir that fixes its temperature.

Without further ado, we take $\Om_{\text{tot}}$ to consist of many ($R \gg 1$) identical \textbf{replicas} of the system \Om that we're interested in.
All of these replicas are in thermal contact with each other, and in thermodynamic equilibrium.\footnote{The thermal contact between any two replicas can be indirect, mediated by a sequence of intermediate replicas.  This transitivity of thermodynamic equilibrium is sometimes called the \href{https://en.wikipedia.org/wiki/Zeroth_law_of_thermodynamics}{zeroth law of thermodynamics}.  It declares that if systems $\Om_A$ \& $\Om_B$ are in thermodynamic equilibrium while systems $\Om_B$ \& $\Om_C$ are in thermodynamic equilibrium, then $\Om_A$ \& $\Om_C$ must also be in thermodynamic equilibrium.}
Choosing one of the replicas to be the system of interest, $\Om$, the other $R - 1 \gg 1$ replicas provide the thermal reservoir $\Om_{\text{res}}$.
Assuming we want to study reasonable systems $\Om$, this ansatz ensures that $\Om_{\text{res}}$ is also reasonable, simply much larger.

An extremely small example of this setup is illustrated by the figures below, where the system of interest consists of only $N = 2$ spins.
For now we assume the spins are \textit{distinguishable}, so that $\downarrow\uparrow$ and $\uparrow\downarrow$ are both distinct micro-states.
This means that each individual replica has only the $M = 4$ micro-states $\om_i$ defined below.
\begin{center}
  \includegraphics[width=0.7\textwidth]{figs/week03_spin-system.pdf}
\end{center}
To form the overall system $\Om_{\text{tot}}$ we now bring together the $R = 9$ replicas shown below.
We draw boxes around each replica to remind us that they are allowed to exchange only energy with each other, while the $N = 2$ spins are conserved in each replica.
We pick out one of these replicas (coloured red) to serve as the system \Om we will consider.
The other $8$ are the thermal reservoir $\Om_{\text{res}}$ that fixes the temperature of $\Om$.
\begin{center}
  \includegraphics[width=0.7\textwidth]{figs/week03_spin-reservoir.pdf}
\end{center}

A convenient way to analyze the overall system of $R$ replicas, $\Om_{\text{tot}}$, is to define the \textbf{occupation number} $n_i$ to be the number of replicas that adopt the micro-state $\om_i \in \Om$ in any given micro-state of $\Om_{\text{tot}}$.
The index $i \in \left\{1, 2, \cdots, M\right\}$ runs over all $M$ micro-states of $\Om$.
In the example above, three of the replicas in the second figure have the micro-state $\om_1 = \downarrow\downarrow$, meaning $n_1 = 3$.
What are the occupation numbers $\left\{n_2, n_3, n_4\right\}$ for the other three $\om_i$ in the figures above?
Are all replicas are accounted for, $\sum_i n_i = R$?
\begin{mdframed}
  \ \\[50 pt]
\end{mdframed}
Normalizing the occupation number by $R$ gives us a well-defined \textit{occupation probability}, $p_i = n_i / R$ with $\sum_i p_i = 1$.
This $p_i$ is the probability that if we choose a replica at random it will be in micro-state $\om_i$.

Now let us consider conservation of energy, which continues to apply to the total energy $E_{\text{tot}}$ of the overall system $\Om_{\text{tot}}$.
We assume that each replica's energy $E_r$ is independent of all the other replicas.
This is guaranteed for the non-interacting systems we will focus on until week $10$, and also holds when interactions are allowed within each replica but not between different replicas.
The thermal contact between replicas allows $E_r$ to fluctuate (subject to conservation of $E_{\text{tot}}$), but there are only $M$ possible values $E_i$ it can have, corresponding to the $M$ micro-states $\om_i \in \Om$.
This allows us to rearrange a sum over replicas into a sum over the micro-states of $\Om$:
\begin{equation}
  \label{eq:canon_Etot}
  E_{\text{tot}} = \sum_{r = 1}^R E_r = \sum_{i = 1}^M n_i E_i,
\end{equation}
with the occupation number $n_i$ counting how many times micro-state $\om_i$ appears among the $R$ replicas.
We can assume that $R$ and $M$ are both finite, so we don't need to worry about rearranging infinite sums.
% ------------------------------------------------------------------



% ------------------------------------------------------------------
\subsubsection{Partition function}
Following Gibbs, we have already taken the thermal reservoir $\Om_{\text{res}}$ to consist of $R - 1$ replicas of the system of interest, $\Om$.
The next step is to further simplify the mathematics by assuming that the overall $R$-replica system $\Om_{\text{tot}}$ is fully specified by a fixed set of $M$ occupation numbers $\left\{n_i\right\}$.
From \eq{eq:canon_Etot}, we see that this ensures conservation of the total energy $E_{\text{tot}}$, and we can apply the micro-canonical tools we developed last week.
Recall our ultimate \textcolor{green}{goal} of showing that such details of the thermal reservoir are irrelevant to the system $\Om$.

Based on the conservation of $E_{\text{tot}}$, we want to determine the (intensive) temperature of $\Om_{\text{tot}}$, which fixes the temperature of the system of interest, $\Om$.
According to our work last week, to do this we first need to compute the overall number of micro-states $M_{\text{tot}}$ as a function of $E_{\text{tot}}$, from which we can derive the entropy and temperature since the system is in thermodynamic equilibrium.
From the fixed occupation numbers $n_i$, we already know how many times each micro-state $\om_i$ appears among the $R$ replicas.
To determine $M_{\text{tot}}$ we just need to count how many possible ways there are of distributing the $\left\{n_i\right\}$ micro-states among the $R$ replicas.

If we consider first the micro-state $\om_1$, the number of possible ways of distributing $n_1$ copies of this micro-states among the $R$ replicas is just the binomial coefficient
\begin{equation*}
  \binom{R}{n_1} = \frac{R!}{n_1! \; (R - n_1)!}.
\end{equation*}
Moving on to $\om_2$, we need to keep in mind that $n_1$ replicas have already been assigned micro-state $\om_1$, so there are only $R - n_1$ replicas left to choose from.
What is the resulting number of possible ways of distributing these $n_2$ micro-states?
\begin{mdframed}
  \ \\[50 pt]
\end{mdframed}
Repeating this process for all micro-states $\left\{\om_1, \om_2, \cdots, \om_M\right\}$, and recalling that $\left(R - \sum_i n_i \right)! = 0! = 1$, you should obtain a product that `telescopes' to
\begin{equation}
  \label{eq:telescoped}
  M_{\text{tot}} = \frac{R!}{n_1! \; n_2! \; \cdots \; n_M!}.
\end{equation}
From this we can see that the order in which we assign micro-states to replicas is irrelevant, since integer multiplication is commutative.

Thanks to thermodynamic equilibrium, the entropy of the micro-canonical overall system $\Om_{\text{tot}}$ is
\begin{equation*}
  S(E_{\text{tot}}) = \log M_{\text{tot}} = \log(R!) - \sum_{i = 1}^M \log(n_i!),
\end{equation*}
where the dependence on $E_{\text{tot}}$ enters through the occupation numbers via \eq{eq:canon_Etot}.
With $R \gg 1$ and $n_i \gg 1$ for all $i = 1, \cdots, M$, we can approximate each of these logarithms using the first two terms in \href{https://en.wikipedia.org/wiki/Stirling's_approximation}{Stirling's formula},
\begin{align*}
  \log(N!) & = N \log N - N + \cO(\log N) \approx N \log N - N &
  \mbox{for } N & \gg 1.
\end{align*}
Back in \eq{eq:CLT_states} we used the central limit theorem to derive a form of this approximation that included a leading term from the $\cO(\log N)$ we neglect here.
In order for \textit{every} occupation number to be large, $n_i \gg 1$, the number of replicas must be much larger than the number of micro-states of $\Om$, so that $R \gg M$.
As we have discussed before, the number of micro-states $M$ is typically a very large number, so we are formally considering truly enormous thermal reservoirs!
This enormity helps ensure that the detailed form of the reservoir will be irrelevant.

Applying the approximation above, what do you find for $S(E_{\text{tot}})$ in terms of $R$ and $n_i$?
What is the entropy in terms of the occupation probabilities $p_i = n_i / R$?
\begin{mdframed}
  $\displaystyle S(E_{\text{tot}}) = \log(R!) - \sum_{i = 1}^M \log(n_i!) \approx $ \\[100 pt]
\end{mdframed}

In your result, the dependence on $E_{\text{tot}}$ now enters through the occupation probabilities $p_i$.
In order to determine the temperature, we have to express $S(E_{\text{tot}})$ directly in terms of $E_{\text{tot}}$.
We do this by applying our knowledge from last week that thermodynamic equilibrium implies maximal entropy.

Following the same steps as last week, we maximize the entropy, including two Lagrange multipliers to account for the two constraints on the occupation probabilities:
\begin{align*}
  \sum_{i = 1}^M p_i & = 1 &
  \sum_{i = 1}^M n_i E_i & = R \sum_{i = 1}^M p_i E_i = E_{\text{tot}}.
\end{align*}
Writing everything in terms of occupation probabilities we therefore need to maximize the modified entropy
\begin{equation*}
  \Sbar = -R \sum_{i = 1}^M p_i \log p_i + \al\left(\sum_{i = 1}^M p_i - 1\right) - \be\left(R \sum_{i = 1}^M p_i E_i - E_{\text{tot}}\right).
\end{equation*}
(The sign of \be is irrelevant, and chosen for later convenience.)
What is the occupation probability $p_k$ that maximizes $\Sbar$?
\begin{mdframed}
  $\displaystyle 0 = \pderiv{\Sbar}{p_k} = $ \\[120 pt] % WARNING: FORMATTING BY HAND
\end{mdframed}

You should find a probability
\begin{equation}
  \label{eq:canon_prob}
  p_k = \frac{1}{Z} e^{-\be E_k},
\end{equation}
where we define $Z = \exp\left[1 - \frac{\al}{R}\right]$ to put the result in its traditional form.
In place of $\left\{\al, \be\right\}$, our free parameters are now $\left\{Z, \be\right\}$.
Still following last week's procedure, we need to fix these free parameters by demanding that the two constraints above are satisfied.
The first of these constraints is straightforward and produces an important result:
\begin{equation}
  \label{eq:part_func}
  1 = \sum_{i = 1}^M p_i = \frac{1}{Z} \sum_{i = 1}^M e^{-\be E_i} \qquad \implies \qquad Z(\be) = \sum_{i = 1}^M e^{-\be E_i}.
\end{equation}

\begin{shaded}
  Equation~\ref{eq:part_func} defines the canonical \textbf{partition function} $Z(\be)$, a fundamental quantity in the canonical ensemble, from which many other derived quantities can be obtained.
\end{shaded}

$Z(\be)$ still depends on the other as-yet-unknown free parameter $\be(E_{\text{tot}})$.
If we apply our second constraint, \eq{eq:canon_Etot}, we can relate \be to $E_{\text{tot}}$:
\begin{equation}
  \label{eq:canon_aveE}
  E_{\text{tot}} = R \sum_{i = 1}^M p_i E_i = \frac{R}{Z(\be)} \sum_{i = 1}^M E_i \; e^{-\be E_i} = R \frac{\sum_{i = 1}^M E_i \; e^{-\be E_i}}{\sum_{i = 1}^M e^{-\be E_i}}.
\end{equation}
This is a bit opaque, but will suffice for our goal of expressing the entropy in terms of $E_{\text{tot}}$.
Inserting \eq{eq:canon_prob} for $p_i$ into your earlier result for the entropy, what do you obtain upon applying Eqs.~\ref{eq:part_func} and \ref{eq:canon_aveE}?
\begin{mdframed}
  $\displaystyle S(E_{\text{tot}}) = -R \sum_{i = 1}^M p_i \log p_i = $ \\[100 pt]
\end{mdframed}
There is a pleasant simplification when we take the derivative to determine the temperature.
Defining $\be' = \pderiv{}{E_{\text{tot}}} \be(E_{\text{tot}})$, we have
\begin{equation*}
  \frac{1}{T} = \pderiv{}{E_{\text{tot}}} S(E_{\text{tot}}) = \pderiv{}{E_{\text{tot}}} \left[E_{\text{tot}} \be + R\log Z(\be)\right] = \be + E_{\text{tot}} \be' + R \frac{1}{Z} \pderiv{Z(\be)}{\be} \be' .
\end{equation*}
Using \eq{eq:canon_aveE} we can recognize
\begin{equation*}
  \frac{1}{Z} \pderiv{Z(\be)}{\be} = \frac{1}{Z} \pderiv{}{\be} \sum_{i = 1}^M e^{-\be E_i} = -\frac{1}{Z} \sum_{i = 1}^M E_i \; e^{-\be E_i} = -\frac{E_{\text{tot}}}{R},
\end{equation*}
so that we don't need to figure out the explicit form of $\be'$:
\begin{equation}
  \label{eq:beta}
  \frac{1}{T} = \be + E_{\text{tot}} \be' - E_{\text{tot}} \be' = \be .
\end{equation}

What's truly remarkable about \eq{eq:beta} is that all the details of the thermal reservoir have vanished---there is no reference to the $R$ replicas or any extensive quantity such as $E_{\text{tot}}$.
This is \textcolor{green}{the goal} we have been pursuing since the start of the notes for this week!
The large thermal reservoir is still present to fix the temperature $T$ characterizing the canonical system $\Om$, but beyond that nothing about it is relevant---or even knowable in the canonical approach.
Every aspect of \Om can now be specified in terms of the fixed temperature $T$ and conserved particle number $N$, starting with the parameter $\be = 1 / T$.

In particular, the partition function from \eq{eq:part_func} is simply
\begin{equation}
  Z(T) = \sum_{i = 1}^M e^{-E_i / T}.
\end{equation}
and together with \be specifies the probabilities
\begin{equation}
  p_i = \frac{1}{Z} e^{-E_i / T}
\end{equation}
from \eq{eq:canon_prob}.
This $p_i$ is now the probability---in thermodynamic equilibrium---that \Om adopts micro-state $\om_i$ with (non-conserved) internal energy $E_i$.
As we might have expected based on last week's look at the micro-canonical ensemble, all micro-statse with the same energy have the same probability in thermodynamic equilibrium.
% ------------------------------------------------------------------



% ------------------------------------------------------------------
\subsection{Internal energy, entropy and heat capacity}
Now that the temperature of the system \Om is fixed, its internal energy is no longer conserved, and can fluctuate by exchanging energy with the thermal reservoir.
Although the internal energy fluctuates, its expectation value $\vev{E}$ is a derived quantity of interest, which can be defined in the usual way for the probability space of the canonical ensemble:
\begin{equation*}
  \vev{E}\!(T) = \sum_{i = 1}^M E_i \; p_i = \frac{1}{Z} \sum_{i = 1}^M E_i e^{-\be E_i}.
\end{equation*}
Here we have explicitly noted the temperature-dependence of $\vev{E}$, and also freely interchanged $\be = 1 / T$, in part because the last expression may \TODO{...}


\TODO{Being written...}
% ------------------------------------------------------------------



% ------------------------------------------------------------------
\newpage % TODO: Placeholder...
\subsection{Helmholtz free energy}
\TODO{Being written...}
% ------------------------------------------------------------------



% ------------------------------------------------------------------
\newpage % TODO: Placeholder...
\subsection{Distinguishable Spins}
\TODO{Being written...}
% ------------------------------------------------------------------



% ------------------------------------------------------------------
\newpage % TODO: Placeholder...
\subsection{Indistinguishable Spins}
\TODO{Being written...}
% ------------------------------------------------------------------


\newpage
% ------------------------------------------------------------------
\renewcommand{\thisweek}{MATH327 Week 4}
\renewcommand{\moddate}{Last modified 25 Jan.~2021}
\setcounter{section}{4}
\section*{Week 4: Ideal gases}
\addcontentsline{toc}{section}{Week 4: Ideal gases}

\TODO{Being written...}
% ------------------------------------------------------------------


\newpage
% ------------------------------------------------------------------
\renewcommand{\thisweek}{MATH327 Week 5}
\renewcommand{\moddate}{Last modified 25 Jan.~2021}
\setcounter{section}{5}
\setcounter{subsection}{0}
\phantomsection
\addcontentsline{toc}{section}{Week 5: Thermodynamic cycles}
\section*{Week 5: Thermodynamic cycles}

\TODO{Being written...}
% ------------------------------------------------------------------


\newpage
% ------------------------------------------------------------------
\renewcommand{\thisweek}{MATH327 Week 6}
\renewcommand{\moddate}{Last modified 3 Feb.~2021}
\setcounter{section}{6}
\setcounter{subsection}{0}
\phantomsection
\addcontentsline{toc}{section}{Week 6: Grand-canonical ensemble}
\section*{Week 6: Grand-canonical ensemble}
\subsection{The particle reservoir}
\TODO{Being written...}
% ------------------------------------------------------------------



% ------------------------------------------------------------------
\newpage % TODO: Placeholder...
\subsection{The grand-canonical partition function}
\TODO{Being written...}
% ------------------------------------------------------------------



% ------------------------------------------------------------------
\newpage % TODO: Placeholder...
\subsection{The grand-canonical potential}
\TODO{Being written...}
% ------------------------------------------------------------------


\newpage
% ------------------------------------------------------------------
\renewcommand{\thisweek}{MATH327 Week 7}
\renewcommand{\moddate}{Last modified 10 Apr.~2021}
\setcounter{section}{7}
\setcounter{subsection}{0}
\phantomsection
\addcontentsline{toc}{section}{Week 7: Quantum statistics}
\section*{Week 7: Quantum statistics}
\subsection{Quantized energy levels and their micro-states}
This week we begin applying the grand-canonical ensemble to investigate quantum statistical systems.
The first step is to introduce quantum statistics itself, building on the initial glimpse that we got in \secref{sec:regulate}.
It is worth reiterating that no prior knowledge of quantum physics is assumed, nor will this module attempt to teach quantum mechanics.
We will simply consider quantum behaviour as an ansatz (that turns out to be realized in nature), and analyze the resulting systems by making use of the statistical physics tools we have developed.

Looking back to our derivation of the canonical partition function for a classical (that is, non-quantum) ideal gas in \secref{sec:regulate}, we can recall that we engaged in slightly circular argumentation.
First, because the partition function is defined as a sum over micro-states $\om_i$,
\begin{equation*}
  Z = \sum_i e^{-E(\vec{p}_i) / T},
\end{equation*}
we had to conjecture that the gas particles' momenta $\vec{p}_i$ are \textit{quantized} and can take only particular discrete values, rather than varying continuously.
These quantized momenta produce a countable number of discrete \textit{energy levels}, leading to a countable number of micro-states and hence a well-defined partition function that takes the form of a sum over all possible discrete momenta for each particle.
Second, we then assumed that the energy levels are spaced very close to each other, allowing us to approximate that sum as a multi-dimensional gaussian integral.
That is, we went right back to working with continuously varying momenta, despite the formal need to regulate the system by quantization.

For the next few weeks, we will work in the quantum regime where all energy levels of a system remain discrete. % TODO: Could talk about temperature setting an energy scale, but that might be a distraction here and can be pointed out later on...
In addition, a more subtle change of approach is required by the fundamental indistinguishability of particles governed by quantum mechanics.
This quantum indistinguishability is a fact about nature that we will take as given.

To appreciate the consequences of quantum indistinguishability, let's first apply our usual (classical) approach to compute the grand-canonical partition function for a system with discrete energy levels $E_{\ell}$ for $\ell = 0$, $1$, $\cdots$, $L$.\footnote{The countable energy levels mean the system remains well defined even for $L \to \infty$.}
%Without loss of generality, we can take $E_{\ell} \geq 0$
Despite the discrete energy levels, this calculation will produce a non-quantum result known as \textbf{Maxwell--Boltzmann} (MB) statistics (named after \href{https://en.wikipedia.org/wiki/James_Clerk_Maxwell}{James Clerk Maxwell} and Ludwig Boltzmann).
We will be able to see when this result is a good approximation and when it breaks down.

Starting from the general expression for the grand-canonical partition function, \eq{eq:grand_part_func},
\begin{equation*}
  Z_g(\be, \mu) = \sum_i e^{-\be (E_i - \mu N_i)},
\end{equation*}
we just need to define the micro-states $\om_i$ with energy $E_i$ and particle number $N_i$.
In the classical Maxwell--Boltzmann approach, we first sum over all possible particle numbers,
\begin{equation*}
  \ZMB(\be, \mu) = \sum_{i, N_i = 0} e^{-\be E_i} + \sum_{j, N_j = 1} e^{-\be (E_j - \mu)} + \sum_{k, N_k = 2} e^{-\be (E_k - 2\mu)} + \cdots,
\end{equation*}
where the micro-states labelled $\left\{\om_i, \om_j, \om_k, \cdots\right\}$ are those that have $N = 0$, $1$, $2$, $\cdots$ particles, respectively.
We can recognize $N$-particle canonical partition functions $Z_N(\be)$ in the expression above,
\begin{equation}
  \label{eq:fugacity_exp}
  \ZMB(\be, \mu) = Z_0(\be) + e^{\be\mu} Z_1(\be) + e^{2\be\mu} Z_2(\be) + \cdots = \sum_{N = 0}^{\infty} \left[e^{\be\mu}\right]^N Z_N(\be),
\end{equation}
allowing us to benefit from our experience with the canonical ensemble.
(This is a general result known as the \textit{fugacity expansion}, where $e^{\be\mu}$ is called the fugacity.)

In particular, because we continue to consider only `ideal' systems in which the particles don't interact with each other, each $Z_N(\be)$ is simply the product of the single-particle partition functions $Z_1(\be)$ for all $N$ independent particles,
\begin{equation*}
  Z_N(\be) = \frac{1}{N!} \left[Z_1(\be)\right]^N,
\end{equation*}
with the factor of $N!$ included to correct for over-counting indistinguishable particles.
This is exactly the derivation we performed in \secref{sec:regulate}, to obtain \eq{eq:ideal_indis} for the classical ideal gas.
Inserting this into \eq{eq:fugacity_exp}, we have
\begin{equation*}
  \ZMB(\be, \mu) = \sum_{N = 0}^{\infty} \frac{1}{N!} \left[e^{\be\mu}\right]^N \left[Z_1(\be)\right]^N = \sum_{N = 0}^{\infty} \frac{1}{N!} \left[e^{\be\mu} Z_1(\be)\right]^N = \exp\left[e^{\be\mu} Z_1(\be)\right].
\end{equation*}
In the case of a system with discrete energy levels $E_{\ell}$, the single-particle partition function is simply
\begin{equation*}
  Z_1(\be) = \sum_{\ell = 0}^L e^{-\be E_{\ell}},
\end{equation*}
leading to the Maxwell--Boltzmann grand-canonical partition function
\begin{equation}
  \label{eq:partfunc_MB}
  \ZMB(\be, \mu) = \exp\left[e^{\be\mu} \sum_{\ell = 0}^L e^{-\be E_{\ell}} \right] = \exp\left[\sum_{\ell = 0}^L e^{-\be\left(E_{\ell} - \mu\right)}\right].
\end{equation}

Unfortunately, as mentioned in a footnote accompanying \eq{eq:ideal_indis}, this derivation relies on the assumption that every particle occupies a different energy level.
While this would be effectively guaranteed when the particles' energies vary continuously, and can be an excellent approximation when there are many energy levels spaced very close to each other, the assumption breaks down if there is a non-negligible chance of two particles occupying the same energy level.

We can illustrate this with a simple exercise of considering a system with $N = 2$ particles that can occupy any of five energy levels.
For a further simplification, let's suppose that all five energy levels have the same value of the energy, namely $E_0 = E_1 = E_2 = E_3 = E_4 = 0$.
(Such distinct energy levels that have the same value of the energy are said to be \textit{degenerate}.)
With $E_{\ell} = 0$ for all $\ell$, the canonical partition function simply counts the number of microstates, for example
\begin{equation*}
  Z_1 = \sum_{\ell = 0}^4 e^{-\be E_{\ell}} = \sum_{\ell = 0}^4 1 = 5
\end{equation*}
for all $\be = 1 / T$.
The mathematics is the same as counting the number of ways two balls can be placed in five boxes, with possible micro-states that can be represented as $\boxzero\boxone\boxzero\boxzero\boxone$ and $\boxzero\boxzero\boxtwo\boxzero\boxzero$.
What is the two-particle partition function if the balls are distinguishable?
\begin{mdframed}
  $Z_D = $ \\[24 pt]
\end{mdframed}
For indistinguishable particles, our derivation above would predict the partition function $Z_I = \frac{1}{2} Z_D$, which is not an integer and therefore cannot be correct.

We can spot the error by explicitly writing down all micro-states in both cases of distinguishable and indistinguishable particles.
In the distinguishable case, we can suppose that the balls are red ($\textcolor{red}{\bullet}$) and blue ($\textcolor{blue}{\bullet}$), and compactly label micro-states by recording whether each box is empty (``$0$''), contains the red ball (``$R$''), the blue ball (``$B$'') or both balls (``$2$''):
\begin{align*}
  \boxzero\boxzero\boxed{\textcolor{red}{\bullet}}\boxzero\boxed{\textcolor{blue}{\bullet}} & = 00R0B &
  \boxzero\boxzero\boxed{\textcolor{red}{\bullet}\textcolor{blue}{\bullet}}\boxzero\boxzero & = 00200.
\end{align*}
The full catalog of micro-states is then
\begin{align*}
  RB000 & & 0R0B0 & & BR000 & & 0B0R0 & & 20000 \\
  R0B00 & & 0R00B & & B0R00 & & 0B00R & & 02000 \\
  R00B0 & & 00RB0 & & B00R0 & & 00BR0 & & 00200 \\
  R000B & & 00R0B & & B000R & & 00B0R & & 00020 \\
  0RB00 & & 000RB & & 0BR00 & & 000BR & & 00002
\end{align*}
If we now consider indistinguishable particles where we can only know the number $R = B = 1$, we see that the third and fourth columns above duplicate the first two columns.
This is exactly the over-counting that the usual factor of $\frac{1}{N!} = \frac{1}{2}$ corrects, which leaves us with the micro-states
\begin{align*}
  11000 & & 01010 & & 20000 \\
  10100 & & 01001 & & 02000 \\
  10010 & & 00110 & & 00200 \\
  10001 & & 00101 & & 00020 \\
  01100 & & 00011 & & 00002
\end{align*}
But we see that the micro-states in the final column, with both particles in the same energy level, were not over-counted, and must not be divided by $N!$.

In order to generalize this simple exercise, we note that the micro-states for indistinguishable particles can be systematically labelled by \textit{occupation numbers} $n_{\ell}$, similar to those that we encountered when using replicas to derive the canonical partition function in \secref{sec:reservoir} and the grand-canonical partition function in \secref{sec:Zg}.
Here the occupation number $n_{\ell}$ is simply the number of particles in energy level $E_{\ell}$.
This change of perspective is all we need to define quantum statistics as opposed to classical statistics.

\begin{shaded}
  In \textbf{quantum statistics}, the micro-states are defined by considering each energy level $E_{\ell}$ in turn, and summing over the possible occupation numbers $n_{\ell}$ that it could have.
  This contrasts with the classical approach in which we define the micro-states by considering each particle in turn, and summing over the possible energies $E_{\ell}$ it could have.
\end{shaded}

Recalling the fundamental indistinguishability of particles governed by quantum mechanics, we have seen that the classical approach over-counts micro-states, but this over-counting depends on how likely it is for multiple particles to occupy the same energy level.
The quantum approach of summing over the occupation numbers of the quantized energy levels avoids this issue, and requires no additional factors to correct over-counting.
% ------------------------------------------------------------------



% ------------------------------------------------------------------
\subsection{Bosons and fermions}
In the two sections below we will carry out explicit computations to clarify what the above definition of quantum statistics means in practice.
First, there is one more fact about nature that we need to mention.
This concerns the occupation numbers $n_{\ell}$ that are possible for each energy level $E_{\ell}$.

\href{https://en.wikipedia.org/wiki/Spin-statistics_theorem}{A theorem}, based on quantum mechanics and special relativity, states that all particles are either \textit{bosons} (named after \href{https://en.wikipedia.org/wiki/Satyendra_Nath_Bose}{Satyendra Nath Bose}) or \textit{fermions} (named after \href{https://en.wikipedia.org/wiki/Enrico_Fermi}{Enrico Fermi}).\footnote{The proof assumes three-dimensional physical space, and \href{https://en.wikipedia.org/wiki/Anyon}{more exotic behaviour} is possible for particles confined to two-dimensional surfaces.}
These two classes of particles obey different rules for their possible occupation numbers, and therefore give rise to distinct quantum statistics.

Any non-negative number of identical bosons can simultaneously occupy the same energy level, corresponding to occupation numbers $n_{\ell} = 0$, $1$, $2$, $\cdots$.
Physical examples of bosons include photons (particles of light), pions, helium-$4$ atoms and the famous Higgs particle.

On the other hand, it is impossible for multiple identical fermions to occupy the same energy level, meaning that their only possible occupation numbers are $n_{\ell} = 0$ and $1$.
Physical examples of fermions include electrons, protons, neutrons, neutrinos and helium-$3$ atoms.
This behaviour is known as the \textit{Pauli exclusion principle} (named after \href{https://en.wikipedia.org/wiki/Wolfgang_Pauli}{Wolfgang Pauli}) and has extremely important consequences, including the existence of chemistry and life.

The reason multiple identical fermions cannot occupy the same energy level is due to a feature of quantum mechanics, and not because they physically repel each other.
This paragraph will imprecisely describe that aspect of quantum mechanics for the curious, and can be skipped without any problem.
Consider a system of identical quantum particles occupying various energy levels.
Loosely speaking, if we interchange any pair of those particles, we end up with the same system, up to a factor of $\pm 1$.
Bosons correspond to the intuitive $+1$ case, where interchanging indistinguishable particles has no effect.
Fermions correspond to the unintuitive $-1$ case (which is related to \href{https://en.wikipedia.org/wiki/Grassmann_number}{Grassmann numbers} named after \href{https://en.wikipedia.org/wiki/Hermann_Grassmann}{Hermann Grassmann}).
In this case, if the particles we interchange are occupying the same energy level, the resulting system is exactly the same as the starting point --- but it also has to differ by this factor of $-1$.
Since no non-zero system of particles can equal its negative, no systems with multiple identical fermions in the same energy level can possibly exist.

Looking back at the example system of $N = 2$ particles with five energy levels in the previous section, all $15$ micro-states we wrote down are possible if the particles are bosons.
%If the particles are fermions, the five micro-states in the third column are all forbidden since each of them involves a doubly occupied energy level with $n_{\ell} = 2$.
%Only the other $10$ states with $n_{\ell} \in \left\{0, 1\right\}$ are allowed.
Which of those micro-states are allowed if the particles are fermions?
\begin{mdframed}
  \ \\[24 pt]
\end{mdframed}
This difference in the possible micro-states ensures that bosons and fermions exhibit different quantum statistics.
We will now consider each case in turn.
% ------------------------------------------------------------------



% ------------------------------------------------------------------
\subsection{Bose--Einstein statistics}
The quantum statistics of bosons is known as \textbf{Bose--Einstein} (BE) statistics, named after Satyendra Nath Bose and Albert Einstein.
As described above, to compute the grand-canonical partition function in this case, we first sum over all energy levels,
\begin{equation*}
  \ZBE(\be, \mu) = \TODO{...}
\end{equation*}

\TODO{Being written...}
% ------------------------------------------------------------------



% ------------------------------------------------------------------
\newpage
\subsection{Fermi--Dirac statistics}
The quantum statistics of bosons is known as \textbf{Fermi--Dirac} (FD) statistics, named after Enrico Fermi and \href{https://en.wikipedia.org/wiki/Paul_Dirac}{Paul Dirac}.

\TODO{Being written...}
% ------------------------------------------------------------------


\newpage
% ------------------------------------------------------------------
\renewcommand{\thisweek}{MATH327 Week 8}
\renewcommand{\moddate}{Last modified 17 Apr.~2021}
\setcounter{section}{8}
\setcounter{subsection}{0}
\phantomsection
\addcontentsline{toc}{section}{Week 8: Quantum gases of bosons}
\section*{Week 8: Quantum gases of bosons}
\subsection{The photon gas}
Last week we derived the grand-canonical partition function (\eq{eq:partfunc_BE}) that defines quantum Bose--Einstein statistics for systems of non-interacting bosons,
\begin{equation*}
  \ZBE(\be, \mu) = \prod_{\ell = 0}^L \frac{1}{1 - e^{-\be (E_{\ell} - \mu)}}.
\end{equation*}
This expression results from summing over the possible occupation numbers $n_{\ell} \in \Nbb_0$ for each energy level $\cE_{\ell}$ with energy $E_{\ell}$.
The corresponding grand-canonical potential is
\begin{equation*}
  \Phi_{\text{BE}} = -T \log Z_g = T \sum_{\ell = 0}^L \log\left[1 - e^{-\be (E_{\ell} - \mu)}\right],
\end{equation*}
from which we can determine the large-scale properties of the system, including its average internal energy $\vev{E}$, average particle number $\vev{N}$, entropy $S$, and pressure $P$.

To do so, we have to specify the energy levels of the particles that compose the system of interest, and the degeneracies of those energy levels.
One example of this that we have already seen is the analysis of non-relativistic ideal gas particles in \secref{sec:regulate}.
For a single particle with mass $m$ in a volume $V = L^3$, we determined the quantized energies
\begin{equation}
  \label{eq:nonrel_energy}
  E(k_x, k_y, k_z) = \frac{\hbar^2 \pi^2}{2mL^2}\left(k_x^2 + k_y^2 + k_z^2\right),
\end{equation}
where the integers $k_{x, y, z}$ specify the possible momenta of the particle,
\begin{align*}
  \vec p & = (p_x, p_y, p_z) = \hbar \frac{\pi}{L} (k_x, k_y, k_z) &
  k_{x, y, z} = 1, 2, \cdots.
\end{align*}
(For technical reasons, quantum mechanics requires $k_{x, y, z} \geq 1$, leading us to adjust our ansatz compared to \eq{eq:quant_mom}.)
This system has a unique ground state $\cE_0$ with $\vec k = (1, 1, 1)$ and energy $E_0 = \frac{3}{2} \frac{\hbar^2 \pi^2}{mL^2}$.
The next three energy levels are degenerate, with energy $3 \frac{\hbar^2 \pi^2}{mL^2}$ corresponding to $\vec k = (2, 1, 1)$ and permutations, followed by another three degenerate energy levels with energy $\frac{9}{2} \frac{\hbar^2 \pi^2}{mL^2}$ corresponding to $\vec k = (2, 2, 1)$ and permutations.

This week we will build on that experience to consider a gas of \textit{photons}, massless bosonic quantum particles of light.
For our purposes, with no prior knowledge of particle physics, we can define photons simply by specifying two relevant details of their energy levels.
First, a photon's energy is proportional to the magnitude of its momentum, %with $m = 0$ for massless photons, \eq{eq:nonrel_energy} is clearly problematic.
\begin{equation*}
  \Eph(p) = c \sqrt{p_x^2 + p_y^2 + p_z^2} \equiv c p.
\end{equation*}
Here the speed of light $c$ is really just a unit conversion factor that we could set to $c = 1$ by using appropriate units.
Second, for each momentum $\vec p$, a photon has two degenerate energy levels with the same energy $E(p)$. % TODO: Could mention polarization...

In a volume $V = L^3$, only the same discrete momenta as above are allowed,
\begin{align*}
  p & = \hbar \frac{\pi}{L} \sqrt{k_x^2 + k_y^2 + k_z^2} \equiv \hbar \frac{\pi}{L} k &
  k_{x, y, z} = 1, 2, \cdots,
\end{align*}
so that the quantized photon energies are
\begin{equation}
  \label{eq:photon_Ek}
  \Eph(k) = \hbar c \frac{\pi}{L} k.
\end{equation}
It is conventional to use the speed of light to work with photons in terms of their wavelength \la and angular frequency $\om = 2\pi f$ (not to be confused with generic micro-states $\om_i$), given the relation
\begin{equation*}
  c = \frac{\la \om}{2\pi}.
\end{equation*}
Just like the momenta, the wavelengths \la are also quantized in volume $V = L^3$,
\begin{equation*}
  \la = \frac{2L}{k} \qquad \Lra \qquad c = \frac{\om}{\frac{\pi}{L} k},
\end{equation*}
and we can rewrite \eq{eq:photon_Ek} as
\begin{equation}
  \label{eq:photon_omega}
  \Eph(\om) = \hbar \om.
\end{equation}
Low (\textit{infrared}) frequencies correspond to small energies and long wavelengths, while high (\textit{ultraviolet}) frequencies correspond to large energies and short wavelengths.

We are now ready to write down the grand-canonical potential for a photon gas:
\begin{equation*}
  \Phi_{\text{ph}} = T \sum_{\ell = 0}^L \log\left[1 - e^{-\be (E_{\ell} - \mu)}\right] = 2T \sum_{\vec k} \log\left[1 - e^{-\be (\Eph(k) - \mu)}\right],
\end{equation*}
where the factor of $2$ in the final expression accounts for the doubly degenerate energy levels.
We can simplify this expression by appreciating that photons are easy to create and destroy.
Every time a light is switched on, it begins emitting a constant flood of photons (with wavelengths of several hundred nanometres).
Food in a microwave oven gets hot by absorbing many lower-energy photons (with longer wavelengths around $12$~centimetres).
In both cases an enormous number of photons is required to make even a small change in energy, so that \eq{eq:mu_E} implies the chemical potential of a photon gas must be negligible,
\begin{equation*}
  \mu = \left.\pderiv{E}{N}\right|_S \approx 0 \qquad \Lra \qquad \Phi_{\text{ph}} \approx 2T \sum_{\vec k} \log\left[1 - e^{-\be \Eph(k)}\right].
\end{equation*}

Another simplification comes from considering the photon gas in a large volume, so that we can approximate the sum over discrete integer $k_{x, y, z}$ by integrals over continuous real $\khat_{x, y, z}$,
\begin{equation*}
  \Phi_{\text{ph}} \approx 2T \int d\khat_x d\khat_y d\khat_z \log\left[1 - e^{-\be \Eph(\khat)}\right].
\end{equation*}
Since the energy $\Eph(\khat)$ depends only on the magnitude $\khat$, we can profit from converting to spherical coordinates.
When we do so, we have to keep in mind that $k_{x, y, z} > 0$ corresponds only to the positive octant of thee sphere,
\begin{equation*}
  \int_0^{\infty} d\khat_x \int_0^{\infty} d\khat_y \int_0^{\infty} d\khat_z = \int_0^{\infty} d\khat \; \khat^2 \int_0^{\pi / 2} d\theta \; \sin\theta \int_0^{\pi / 2} d\phi = \frac{\pi}{2} \int_0^{\infty} d\khat \; \khat^2,
\end{equation*}
so that
\begin{equation*}
  \Phi_{\text{ph}} \approx \pi T \int_0^{\infty} d\khat \; \khat^2 \log\left[1 - e^{-\be \Eph(\khat)}\right].
\end{equation*}
We can finally change variables to integrate over the photon angular frequency $\om = c \frac{\pi}{L} k$, with $\Eph = \hbar \om$, to find
\begin{align}
  \Phi_{\text{ph}} & \approx \pi T \left(\frac{L}{\pi c}\right)^3 \int_0^{\infty} d\om \; \om^2 \log\left[1 - e^{-\be \hbar \om}\right] \cr
                   & = \frac{VT}{c^3 \pi^2} \int_0^{\infty} d\om \; \om^2 \log\left[1 - e^{-\be \hbar \om}\right], \label{eq:photon_grand}
\end{align}
recognizing $L^3 = V$.
With this grand-canonical potential derived, we just need to take the appropriate derivatives to determine the thermodynamics and equation of state for the photon gas.
% ------------------------------------------------------------------



% ------------------------------------------------------------------
\subsection{The sun and the void}
We are now ready to analyze the average internal energy from the grand-canonical potential for a photon gas, \eq{eq:photon_grand}.
With $\mu = 0$, \eq{eq:E_grand} from week~6 becomes
\begin{equation*}
  \vev{E}_{\text{ph}} = -T^2 \pderiv{}{T} \left[\frac{\Phi_{\text{ph}}}{T}\right] = \pderiv{}{\be} \left[\be \Phi_{\text{ph}}\right].
\end{equation*}
To begin, we will consider the energy density expressed as an integral over photon frequencies,
\begin{equation*}
  \frac{\vev{E}_{\text{ph}}}{V} = \int_0^{\infty} P(\om) \; d\om,
\end{equation*}
where the function $P(\om)$ is known as the \textit{spectral density}, or simply the \textit{spectrum}.
What is the spectrum for a photon gas?
\begin{mdframed}
  $\displaystyle \frac{\vev{E}_{\text{ph}}}{V} = \frac{1}{c^3 \pi^2} \int_0^{\infty} d\om \; \om^2 \pderiv{}{\be} \log\left[1 - e^{-\be \hbar \om}\right] = $ \\[100 pt]
\end{mdframed}

You should find
\begin{equation}
  \label{eq:Planck_omega}
  P(\om) = \left(\frac{\hbar}{c^3 \pi^2}\right) \frac{\om^3}{e^{\be \hbar \om} - 1},
\end{equation}
which is known as the Planck spectrum, named after Max Planck.
The Planck spectrum is plotted in the figure below, which comes from \href{https://commons.wikimedia.org/wiki/File:Black_body.svg}{Wikimedia Commons}.

\begin{center}\includegraphics[width=\textwidth]{figs/week08_spectrum.pdf}\end{center}

In this plot the horizontal axis uses the wavelength $\la = 2\pi c / \om$.
Changing variables in your work above, what is Planck spectrum $P(\la)$ as a function of wavelength?
\begin{mdframed}
  $\displaystyle \frac{\vev{E}_{\text{ph}}}{V} = \frac{\hbar}{c^3 \pi^2} \int_0^{\infty} \frac{\om^3}{e^{\be \hbar \om} - 1} \; d\om = $ \\[120 pt] % WARNING: FORMATTING BY HAND
\end{mdframed}

You should find
\begin{equation}
  \label{eq:Planck_la}
  P(\la) = \left(\frac{16\pi^2 \hbar c}{\la^5}\right) \frac{1}{e^{2\pi\be \hbar c / \la} - 1},
\end{equation}
which is plotted\footnote{The plot divides our $P(\la)$ by $4\pi$~steradian to consider the spectrum per unit of solid angle.} for three temperatures $T = 1 / \be$ in the figure above.
Considering first the high-energy ultraviolet (UV) limit of small wavelengths $\la$, we can see from \eq{eq:Planck_la} that $P(\la)$ is exponentially suppressed, which overwhelms the diverging factor $\propto 1 / \la^5$ in parentheses.

In the low-energy infrared limit, the large $\la$ has the same effect that a large temperature ($\be \ll 1$) would have: $e^{2\pi\be \hbar c / \la} - 1 \approx 2\pi\be \hbar c / \la$ and
\begin{equation*}
  P(\la) \approx \left(\frac{16\pi^2 \hbar c}{\la^5}\right) \frac{\la}{2\pi\be \hbar c} = \frac{8\pi T}{\la^4}.
\end{equation*}
The connection to large temperatures indicates that this is the result classical statistics would predict for the energy spectrum of light.
It is known as the Rayleigh--Jeans spectrum (named after \href{https://en.wikipedia.org/wiki/John_William_Strutt,_3rd_Baron_Rayleigh}{the third Baron Rayleigh} and \href{https://en.wikipedia.org/wiki/James_Jeans}{James Jeans}), and clearly cannot be correct in the ultraviolet limit $\la \to 0$, where it predicts short-wavelength light would possess a diverging amount of energy.
This classical result is known as the \textit{ultraviolet catastrophe}, and Planck's (heuristic) solution to it was one of the first steps towards quantum physics.


As the temperature increases, the maximum of the Planck spectrum moves to shorter wavelengths and correspondingly larger energies.
The fact that the peak of the spectrum for $T \approx 5000$~K falls around the wavelengths of visible light (roughly $400$--$700$~nm) is not a coincidence.
As shown in the figure below, the 

\begin{center}\includegraphics[width=\textwidth]{figs/week08_sun.pdf}\end{center}



\TODO{Being written...}
% ------------------------------------------------------------------



% ------------------------------------------------------------------
\newpage
\subsection{Photon gas equation of state}
\TODO{Being written...}
% ------------------------------------------------------------------



% ------------------------------------------------------------------
\newpage
\subsection{Bose--Einstein condensation}
\TODO{Being written...} % BEC now routinely produced in hundreds of labs around the world, including as tool for quantum computing (e.g., arXiv:2003.08945)...
% ------------------------------------------------------------------


\newpage
% ------------------------------------------------------------------
\renewcommand{\thisweek}{MATH327 Week 9}
\renewcommand{\moddate}{Last modified 25 Jan.~2021}
\setcounter{section}{9}
\section*{Week 9: Quantum gases of fermions}
\addcontentsline{toc}{section}{Week 9: Quantum gases of fermions}

\TODO{Being written...}
% ------------------------------------------------------------------


\newpage
% ------------------------------------------------------------------
\renewcommand{\thisweek}{MATH327 Week 10}
\renewcommand{\moddate}{Last modified 1 May 2021}
\setcounter{section}{10}
\setcounter{subsection}{0}
\phantomsection
\addcontentsline{toc}{section}{Week 10: Interacting systems}
\section*{Week 10: Interacting systems}
\subsection{From non-interacting spins to the Ising model}
So far in this module we have considered `ideal' systems whose constituent objects do not interact with each other.
While we have seen that excellent mathematical models for real physical systems (such as stars and the cosmic microwave background) can be obtained despite this approximation of non-interacting particles, there are important statistical physics phenomena that cannot be captured by this approach.

An important class of examples, which we will investigate this week, are \textbf{phase transitions}, where interactions allow the same particles to produce extremely different large-scale behaviours, depending on control parameters such as the temperature or pressure.
An everyday example is the transition of H$_2$O molecules from liquid water to solid ice as the temperature decreases.
As the temperature of the universe itself decreased during the first few micro-seconds following the big bang, elementary particles transitioned from a so-called quark--gluon plasma to the protons and neutrons we are made out of today.
An intermediate example illustrated in the figure below (\href{https://doi.org/10.1063/PT.3.4384}{source}) involves two layers of graphene at a low temperature $T \approx 1.7$~K.
If these two layers are rotated with respect to each other by a small ``magic angle'' $\theta \approx 1.1^{\circ}$, the system transitions from being an electrical insulator to being a superconductor.

\begin{center}\includegraphics[width=0.5\textwidth]{figs/week10_graphene.pdf}\end{center}

We will introduce interactions and explore their effects using simple spin systems of the sort we previously analyzed in some depth during weeks $2$ and $3$.
In the non-interacting case we previously considered, the internal energy of the system (\eq{eq:spin_energy}) is
\begin{equation*}
  E = H \sum_{n = 1}^N s_n \qquad \mbox{(non-interacting)},
\end{equation*}
where $H > 0$ is the constant strength of an external magnetic field and the orientation of the $n$th spin, $s_n$, takes one of only two possible values: $s_n = 1$ if the spin is aligned anti-parallel to the field and $s_n = -1$ if the spin is aligned parallel to the field.
The ground state of the system features all $N$ spins aligned parallel to the magnetic field, with minimal energy $E_0 = -NH$.
This week we will only consider systems of distinguishable spins, which can be labeled by their fixed position in a $d$-dimensional simple cubic \textbf{lattice} like that shown below for $d = 2$ dimensions.
The $d = 1$ case of a one-dimensional lattice is precisely the system of spins arranged in a line that we analyzed in \secref{sec:spin_chain}.

\begin{center}\includegraphics[width=0.8\textwidth]{figs/week10_spins.pdf}\end{center}

We can see that the total internal energy of the non-interacting system can easily be written as a sum over energies for each individual spin,
\begin{align*}
  E_n & = H s_n &
  E & = \sum_{n = 1}^N E_n \qquad \mbox{(non-interacting)}.
\end{align*}
This is a generic feature of non-interacting systems, and an aspect of the \textbf{factorization} that enormously simplifies calculations by causing the $N$-particle partition function (\eq{eq:spin_part_func}) to take the form of a product of $N$ identical terms, $Z = \left[2\cosh\left(\be H\right)\right]^N = Z_1^N$.
However, a stronger condition needs to be satisfied in order for such factorization to be guaranteed, which rigorously defines what it means for a system to be non-interacting.

\begin{shaded}
  Let $\De E_i$ be the change in the system's internal energy caused by changing its $i$th degree of freedom.
  Then the system is defined to be \textbf{non-interacting} if and only if $\De E_i$ is independent of all other degrees of freedom $k \ne i$.
\end{shaded}

For our system of $N$ distinguishable spins, the only possible change we can make to a degree of freedom is to negate it, $s_i \to -s_i$, which corresponds to flipping its alignment relative to the external magnetic field.
This spin flip causes the total energy to change,
\begin{equation*}
  E = H \sum_{n = 1}^N s_n = H\left(s_i + \sum_{k \ne i} s_k + \right) \quad \lra \quad H\left(-s_i + \sum_{k \ne i} s_k + \right),
\end{equation*}
corresponding to $\De E_i = -2H s_i$, which is indeed independent of all spins $s_k$ with $k \ne i$.
This simple check confirms that our definition works for the non-interacting spin system under consideration.

Now let's convert this setup into a system of interacting spins by adding the simplest possible two-spin contribution to its energy:
\begin{equation}
  \label{eq:Ising_energy}
  E = -\sum_{(ij)} s_i s_j + H \sum_{n = 1}^N s_n.
\end{equation}
The first sum runs over all pairs of nearest-neighbour spins in the lattice, denoted $(ij)$.
What is the change in energy $\De E_i$ from \eq{eq:Ising_energy} upon negating $s_i \to -s_i$?
Does this indicate an interacting or non-interacting system?
\begin{mdframed}
  \ \\[100 pt]
\end{mdframed}
The pictures below illustrate nearest-neighbour pairs for simple cubic lattices in $d = 2$ and $3$ dimensions, while also introducing some additional lattice terminology.

\begin{center}
  \includegraphics[height=0.4\textwidth]{figs/week10_lattice_2d.pdf}\hfill
  \includegraphics[height=0.4\textwidth]{figs/week10_lattice_3d.pdf}
\end{center}

Instead of drawing up- and down-pointing arrows, these pictures identify the spins with \textit{sites} in the lattice represented as points (or larger filled circles).
In simple cubic lattices, all sites are positioned in a regular grid, separated by a constant distance along each basis vector.
In between nearest-neighbour sites, we can draw \textit{links} as solid lines.
The picture of a two-dimensional lattice on the left highlights the four links (with red hatch marks) that correspond to the four nearest neighbours (circled in red) of a particular site (circled in blue).
While we can only have physical lattices with $d = 1$, $2$ or $3$ in nature, the mathematical construction works just as well for any integer $d \geq 1$.
For $d \geq 2$, an elementary unit of surface area is called a \textit{plaquettes}, while for $d \geq 3$ the elementary unit of volume is called a \textit{cube}.

Computing the energy in \eq{eq:Ising_energy} requires determining all of the nearest-neighbour pairs to be summed in the first term, which is equivalent to all of the links in the lattice, $\ell = (ij)$.
The only potential complication to this task is the need to consider what happens at the edges of the (finite) lattice.
We can avoid this complication by imposing \textbf{periodic boundary conditions}, which add an extra link between each site on the left edge of the lattice and a corresponding site on the right edge (and similarly in all other dimensions).
This is illustrated below for the simple one-dimensional lattice, which has been drawn as a circle to emphasize that all $N$ sites remain separated by a constant distance.
In higher dimensions, periodic boundary conditions produce flat (zero-curvature) $d$-dimensional tori that preserve the simple cubic lattice structure.

\begin{center}\includegraphics[width=0.45\textwidth]{figs/week10_lattice_1d.pdf}\end{center}

With periodic boundary conditions, we can easily see that the $N$-site one-dimensional lattice drawn above has $N$ links.
Each site has two links connecting it to its two nearest neighbours, and each of those links is shared between two sites, so that $\#\ell = 2N / 2 = N$.
Looking back to the two-dimensional lattice drawn farther above, the four links per site produce $\#\ell = 4N / 2 = 2N$.
How many terms are there in the sum $\sum_{(ij)}$ in \eq{eq:Ising_energy} for $N$-site lattices with periodic boundary conditions in $d$ dimensions?
\begin{mdframed}
  \ \\[50 pt]
\end{mdframed}

The energy in \eq{eq:Ising_energy}, with nearest-neighbour spins specified by the underlying simple cubic lattice structure, defines a famous system known as the $d$-dimensional \textbf{Ising model}.
Since the 1960s, the Ising model has been the basis of thousands of scientific studies analyzing everything from ferromagnetism to neural networks to urban segregation.\footnote{For a brief summary, see Charlie Wood, ``\href{https://www.quantamagazine.org/the-cartoon-picture-of-magnets-that-has-transformed-science-20200624/}{The Cartoon Picture of Magnets That Has Transformed Science}'', \textit{Quanta Magazine}, 24 July 2020.}
The model was proposed in 1920 by \href{https://en.wikipedia.org/wiki/Wilhelm_Lenz}{Wilhelm Lenz}, whose PhD student \href{https://en.wikipedia.org/wiki/Ernst_Ising}{Ernst Ising} solved the one-dimensional system as a research project in 1924.
Exactly solving the two-dimensional case (with $H = 0$) took another twenty years, culminating in renowned work by \href{https://en.wikipedia.org/wiki/Lars_Onsager}{Lars Onsager} in 1944.
The three-dimensional Ising model remains an open mathematical question, with no known exact solution.

In this context, `solving' the Ising model means deriving a closed-form expression for its canonical partition function,
\begin{equation*}
  Z(\be, N, H) = \sum_{\left\{s_n\right\}} \exp\left[-\be E(s_n)\right] = \sum_{\left\{s_n\right\}} \exp\left[\be\sum_{(ij)} s_i s_j - \be H \sum_n s_n\right].
\end{equation*}
As in \secref{sec:spin_info}, the partition function sums over all possible spin configurations $\left\{s_n\right\}$, which amounts to a sum of $2^N$ exponential factors for $N$ spins, with $\cO(N)$ terms within each exponential.
Now that the system is interacting, the partition function can no longer be factorized into $N$ identical two-term factors, making it extremely difficult to evaluate.
This is why there is no known exact solution to the three-dimensional Ising model, and it also makes `brute-force' numerical computations impractical.
Even for a system of $N = 1023$ spins (tiny compared to Avogadro's number $\sim 10^{23}$) there would be roughly $2^{1023} \sim 10^{310}$ terms in the partition function, far beyond the capabilities of existing or foreseeable supercomputers.
If we attempt `brute-force' numerical computation of every term in the partition function, even for a system of $N = 1023$ spins we would need to evaluate 
% ------------------------------------------------------------------



% ------------------------------------------------------------------
\subsection{Ising model phases and phase transition}
Similarly to how we analyzed non-interacting spin systems in \secref{sec:spin_info}, we can simplify the Ising model by considering its behaviour in the limits of high and low temperatures.
For another simplification, we will set $H = 0$ in this section, and consider
\begin{align}
  \label{eq:Ising_zero_field}
  E & = -\sum_{(ij)} s_i s_j &
  Z(\be, N) & = \sum_{\left\{s_n\right\}} \exp\left[\be\sum_{(ij)} s_i s_j\right].
\end{align}
We will see that the large-scale behaviour of the Ising model is qualitatively different at high temperatures compared to low temperatures.
In other words, the system exhibits two distinct phases for different temperatures.
This is a necessary but not sufficient condition for there to be a true phase transition---a priori, it is possible for there to be a gradual \textit{crossover} between these two phases, as opposed to a rapid transition.
We will use the Ising model to more rigorously define what exactly constitutes a phase transition, and how this can be distinguished from a crossover.

The Ising model partition function becomes extremely simple in the \textbf{high-temperature} limit $\be \to 0$.
What is its asymptotic limit?
\begin{mdframed}
  $\displaystyle Z(\be = 0, N) = $ \\[50 pt]
\end{mdframed}
You should find a result identical to that for a micro-canonical system with energy $E = 0$, which is clearly non-interacting since $\De E_i = 0$.
Every spin configuration is a different micro-state of the system, all with the same probability $p_i = 1 / 2^N$, as in \eq{eq:micro_equil}.

Effectively, we are considering temperatures so high that the energy from \eq{eq:Ising_zero_field} is negligible for any spin configuration.
Although the energy no longer distinguishes between different micro-states, we can define a quantity that continues to be sensitive to the details of the spin configuration.
This is the \textbf{magnetization} $M = n_+ - n_-$, where we redefine $n_{\pm}$ to be the number of spins with value $\pm 1$, so that $N = n_+ + n_-$.\footnote{In week $2$ we defined $n_{\pm}$ based on spins' alignment with or against the external magnetic field, which no longer applies now that we have set $H = 0$.}
It is convenient to normalize the total magnetization $M$ by the number of spins,
\begin{equation}
  |m| \equiv \frac{|M|}{N} = \frac{|n_+ - n_-|}{n_+ + n_-}
\end{equation}
so that $0 \leq |m| \leq 1$ for any value of $N$.

Our task is now to determine the magnetization of the Ising model at high temperatures.
Above we found that all spin configurations are equally probable in this regime, so $|m|$ will be determined by how likely it is for these micro-states to have a particular magnetization.
For example, there are only two micro-states with $|m| = 1$, corresponding to $(n_+, n_-) = (N, 0)$ and $(0, N)$.
In general, just as we saw in \eq{eq:spin_states}, there are
\begin{equation*}
  \binom{N}{n_+} = \binom{N}{n_-} = \frac{N!}{n_+! \; n_-!}
\end{equation*}
equally probable micro-states with a given $n_+ = N - n_-$.
For large $N \gg 1$ this binomial coefficient is factorially peaked around
\begin{equation*}
  n_+ = n_- = \frac{1}{2} N \qquad \lra \qquad |m| = 0,
\end{equation*}
which defines a \textbf{disordered phase} with similar numbers of up- and down-pointing spins producing a small magnetization.
In the so-called \textit{thermodynamic limit} $N \to \infty$, the magnetization in the disordered phase vanishes exactly, $|m| \to 0$.

We now need to determine the magnetization in the \textbf{low-temperature} limit $\be \to \infty$.
In this regime, as we saw in \secref{sec:spin_chain}, the Boltzmann factor $\exp\left[\be\sum_{(ij)} s_i s_j\right]$ makes it exponentially more likely for the system to adopt micro-states with lower energies.
In particular, we can expect the ground state to determine the magnetization $|m|$ up to exponentially suppressed corrections from higher-energy excited states.
With $H = 0$, the Ising model has two degenerate ground states corresponding to the two ways all the spins can be aligned with each other: $(n_+, n_-) = (N, 0)$ and $(0, N)$.
What is the ground-state energy of the $N$-site Ising model in $d$ dimensions?
\begin{mdframed}
  $\displaystyle E_0 = -\sum_{(ij)} s_i s_j = $ \\[50 pt]
\end{mdframed}

As mentioned above, both of these degenerate ground states have the maximal magnetization $|m| = 1$.
Let's check what effect the first excited state would have on the overall magnetization of the system.
The first excited state involves negating (or `flipping') a single spin, corresponding to $(n_+, n_-) = (N - 1, 1)$ and $(1, N - 1)$.
Because any one of the $N$ spins in the lattice could be flipped, the degeneracy of the first excited state grows with $N$:
\begin{equation*}
  \binom{N}{1} + \binom{N}{N - 1} = 2N.
\end{equation*}
At the same time, as $N$ increases the magnetization of each of these micro-states gets closer to that of the ground state,
\begin{equation*}
  |m| = \frac{N - 1}{N} = 1 - \frac{1}{N}.
\end{equation*}
The key factor is the probability for the system to be in any one of these micro-states, which depends on the energy of the first excited state, $E_1$.
What is the first-excited-state energy of the $N$-site Ising model in $d$ dimensions?
\begin{mdframed}
  $\displaystyle E_1 = $ \\[100 pt]
\end{mdframed}

Let's put things together by computing the relative probability for the $d$-dimensional Ising model to be in its ground state with $|m| = 1$ compared to its first excited state with $|m| = 1 - \frac{1}{N}$, accounting for the different number of micro-states in each case:
\begin{equation*}
  \frac{p(E_0)}{p(E_1)} = \frac{2\cdot \exp\left[\be d\cdot N\right]}{2N\cdot \exp\left[\be \left(d\cdot N - 4d\right)\right]} = \frac{\exp\left[4\be d\right]}{N}.
\end{equation*}
For any fixed $N$, a sufficiently low temperature will cause the ground state to dominate.
This defines an \textbf{ordered phase} in which essentially all spins are aligned in the same direction, producing a large magnetization $|m| = 1$.

We have seen that the magnetization $|m|$ distinguishes between the high- and low-temperature behaviour of the $d$-dimensional Ising model.
In the high-temperature disordered phase, the magnetization is small and $|m| \to 0$ in the thermodynamic limit $N \to \infty$.
In the low-temperature ordered phase, the magnetization is large and $|m| \to 1$ as $T \to 0$.
This is typical behaviour for interacting statistical systems, where the quantity distinguishing between these two phases (here the magnetization) is known as the \textbf{order parameter}.
The behaviour of the order parameter is what distinguishes gradual crossovers from rapid phase transitions.

\begin{shaded}
  A phase transition is characterized by a discontinuity in the order parameter or its derivative(s), in the $N \to \infty$ thermodynamic limit.
  The value(s) of the control parameter(s) at which the discontinuity occurs define the \textit{critical point} corresponding to the transition.
\end{shaded}

For the zero-field ($H = 0$) Ising model, the control parameter is the temperature $T$, and any phase transition would occur at a \textbf{critical temperature} $T_C$.
The sketches below illustrate the most common types of phase transitions.
When the order parameter (OP) itself is discontinuous (shown by a dashed line), the transition is said to be a \textit{first-order} phase transition.
When the order parameter is continuous at $T_C$ but its first derivative is discontinuous, the transition is said to be a \textit{second-order} phase transition.
This naming scheme generalizes to higher-order phase transitions, the most remarkable of which is the infinite-order BKT phase transition (named after \href{https://en.wikipedia.org/wiki/Vadim_Berezinskii}{Vadim Berezinskii}, \href{https://en.wikipedia.org/wiki/J._Michael_Kosterlitz}{J.\ Michael Kosterlitz} and \href{https://en.wikipedia.org/wiki/David_J._Thouless}{David Thouless}), which was awarded the 2016 Nobel Prize in Physics.

\begin{center}\includegraphics[width=0.8\textwidth]{figs/week10_transitions.pdf}\end{center}

Because true discontinuities are only possible with an infinite number of degrees of freedom, it is really the way in which the system approaches the $N \to \infty$ thermodynamic limit that distinguishes crossovers from true phase transitions.
We will conclude this week's consideration of the Ising model by checking its order parameter for discontinuities---explicitly in the simple one-dimensional case, and then through a useful approximation that becomes more reliable as $d$ increases.

\vspace{48 pt}
\begin{center}\textbf{More information about phase transitions \\ and the mean-field approximation will be added over the weekend.}\end{center}
% ------------------------------------------------------------------



% ------------------------------------------------------------------
\newpage
\subsection{Solving the one-dimensional Ising model}
% Dropping one link from periodic BCs leaves things not quite exact for finite $N$, but much more straightforward...
% Section~5.3.1 of David Tong's \href{https://www.damtp.cam.ac.uk/user/tong/statphys.html}{\textit{Lectures on Statistical Physics}} (reference~1 in the list of further reading on page~6) presents the exact calculation.

\TODO{Being written...}
% ------------------------------------------------------------------



% ------------------------------------------------------------------
\newpage
\subsection{The mean-field approximation}
% $\De E_i$ from first gap above...
% TODO: Skip straight to Tong's derivation...

\TODO{Being written...}
% ------------------------------------------------------------------


\newpage
% ------------------------------------------------------------------
\renewcommand{\thisweek}{MATH327 Week 11}
\renewcommand{\moddate}{Last modified 15 Jan.~2021}
\setcounter{section}{11}
\section*{Week 11: Synthesis and broader applications}
\addcontentsline{toc}{section}{Week 11: Synthesis and broader applications}

\TODO{Being updated...}
% ------------------------------------------------------------------

% ------------------------------------------------------------------



% ------------------------------------------------------------------
\end{document}
% ------------------------------------------------------------------
