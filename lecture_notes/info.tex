% ------------------------------------------------------------------
\newcommand{\thisweek}{MATH327 information}
\newcommand{\moddate}{Last modified 28 Mar.~2021}
\setcounter{section}{0}
\phantomsection
\addcontentsline{toc}{section}{Module information and logistics}
\section*{Module information and logistics}

\subsection*{Coordinator}
\begin{description}
  \setlength{\itemsep}{1pt}
  \setlength{\parskip}{0pt}
  \setlength{\parsep}{0pt}
  \item[\qquad] Dr David Schaich
  \item[\qquad] Mathematics Building Room 124 (Theoretical Physics Wing)
  \item[\qquad] \href{mailto:david.schaich@liverpool.ac.uk}{david.schaich@liverpool.ac.uk}
  \item[\qquad] \href{http://www.davidschaich.net}{www.davidschaich.net}
\end{description}
% ------------------------------------------------------------------



% ------------------------------------------------------------------
\subsection*{Weekly schedule}
\textbf{Synchronous sessions} are timetabled at 9:00 on Mondays and 13:00 on Thursdays.

\textbf{Office hours} will take place immediately afterwards, at 10:00 on Mondays and 14:00 on Thursdays.
If these times do not work with your schedule, you can also make an appointment through \href{https://calendly.com/daschaich}{calendly.com/daschaich}.
% ------------------------------------------------------------------



% ------------------------------------------------------------------
\subsection*{Delivery plans and module resources}
Redesigning this module for remote delivery requires some experimentation and improvisation, so you should feel free to make requests and suggestions regarding how the module may best be delivered.
In order to ensure that the amount of material remains manageable, we will cover the same content as last year's edition of the module, rewriting it for remote learning in place of in-person lectures and computer labs.
All resources for the module will be collected at \href{https://liverpool.instructure.com/courses/19478}{its Canvas site}, \\
\centerline{\href{https://liverpool.instructure.com/courses/19478}{liverpool.instructure.com/courses/19478}}

I will aim to minimize the number of announcements sent to you through Canvas.
You have the option to receive email notification of these announcements.
If you choose to turn off these email notifications, you should make sure to check Canvas regularly for announcements.
I will assume announcements have been read within 24 hours.

\subsubsection*{Lecture notes}
We will use the \textit{Notes-driven Asynchronous Delivery} framework designed by the Department of Mathematical Sciences, in which the primary learning materials are the lecture notes you are currently reading.
As you read further, you will encounter \textbf{gaps} in the notes where you will be able to check your understanding by completing some exercises that are intended to be bite-sized.
Feel free to complete these exercises in whatever format you find most convenient, for instance on separate pieces of paper.
While the size of the gap is intended to indicate the rough scale of the exercise, your work may or may not fit depending on your handwriting and solution strategy.

You will also see many numbered equations in the lecture notes.
In general, equations are numbered so that they can easily be referred to in subsequent discussions, derivations or examples.
Therefore a number is not necessarily an indication of importance.
Some particularly important results, definitions and concepts will be highlighted by coloured boxes.
You should ensure you understand the key learning objectives in these coloured boxes.
There is no need to memorize any equations.

\subsubsection*{Synchronous sessions}
These gaps will guide our twice-weekly synchronous sessions using Zoom on Monday mornings and Thursday afternoons.
In addition to discussing any general questions you may have, we will aim to complement the asynchronous material by focusing our attention on any gaps that proved challenging to fill.
In order for these synchronous sessions to remain in \textbf{active learning} mode, you may be asked to describe your own approach to any given gap.
Thinking about them in advance will make this easier.
If you get stuck during the synchronous session, I will aim to ask leading questions rather than lecturing.

\subsubsection*{Video recordings and asynchronous discussions}
To supplement these lecture notes, the Department's delivery framework also involves regular video recordings, expected to add up to around 20 minutes per week.
For each week I plan to provide both an introductory video presenting big-picture context and motivation for the week's topic, followed by a concluding summary of the key results to take away as learning outcomes.
\rem{Since \textbf{synchronous sessions will not be recorded}, to encourage your participation,}
\add{By popular request, \textbf{synchronous sessions will be recorded}.}
I also expect to add supplemental videos addressing any particularly tricky issues that arise in the course of those synchronous sessions (for example, working through gaps that prove challenging).
Supplemental exercises may also be released to respond to particular challenges we encounter.
The Canvas site also provides a \textit{\href{https://liverpool.instructure.com/courses/19478/discussion_topics}{Discussion}} utility where questions can be asked, discussed and answered asynchronously.

\subsubsection*{Term schedule and organization}
The planned schedule for the module is summarized on the first page of these notes.
If necessary to respond to unanticipated challenges, there should be some flexibility to adapt this schedule.
The amount and difficulty of material necessarily fluctuate from week to week, and different material may prove more or less difficult to different students.
Sticking with the same content as last year should ensure that these fluctuations average out over the course of the term.

This content is organized around the concept of \textit{statistical ensembles} introduced by \href{https://en.wikipedia.org/wiki/Josiah_Willard_Gibbs}{J.\ Willard Gibbs} in the early 1900s.
In essence, a statistical ensemble is a mathematical framework for concisely describing the properties of (idealized) physical systems with a large number of independent objects, subject to certain constraints.
After a week studying the mathematical foundations underlying these frameworks, we will learn about the \textit{micro-canonical ensemble} in week 2 and the \textit{canonical ensemble} in week 3.
We will spend weeks 4 and 5 on applications of the canonical ensemble, first considering ideal gases and then investigating thermodynamic cycles.
In week 6 we will see our third and final statistical ensemble, the \textit{grand-canonical ensemble}, which we will apply to various types of quantum gases in weeks 7--9.
Those first nine weeks will all consider systems whose constituent objects do not interact with each other.
We will explore effects of interactions in week 10, which opens up a much broader landscape of applications that we will survey for the remainder of the term (along with some overall synthesis and review to help prepare for the final assessment).

The asynchronous materials for each week will be available through Canvas at least a week in advance.
For example, the notes and videos for week 1 (8--15 February) will be available on or before Sunday, 31 January.
In case subsequent corrections or additions to the notes are needed, the date of last modification will be included at the bottom of each page.

\subsubsection*{Version control and co-creation}
An easy way to check any corrections or additions that may need to be made to these notes is provided by the public GitHub repository where their \LaTeX\ source is maintained under version control: \\
\centerline{\href{https://github.com/daschaich/MATH327_2021}{github.com/daschaich/MATH327\_2021}}
In addition, you are also welcome to use GitHub to report issues with the notes\footnote{Questions about the mathematical content should go to the Canvas discussions.} and create pull requests to address them.
% Captions for the video recordings will also be maintained in this repository.
Such co-creation is entirely optional.
If you are interested in learning more about version control with \texttt{git}, the \href{https://software-carpentry.org}{Software Carpentry} project provides free resources at \\
\centerline{\href{https://swcarpentry.github.io/git-novice/}{swcarpentry.github.io/git-novice/}}

\subsubsection*{Expected background}
You may have seen quantum statistics and quantum gases listed in the module schedule on the first page, so I will emphasize that no prior knowledge of quantum mechanics is assumed.
Any necessary information on this topic will be provided.
%
It is also not necessary to be familiar with statistical analysis or combinatorics.
In those contexts I just anticipate that you have previously seen the \href{https://en.wikipedia.org/wiki/Standard_deviation}{standard deviation} and the \href{https://en.wikipedia.org/wiki/Binomial_coefficient}{binomial coefficient}
\begin{equation*}
  \binom{N}{k} = \frac{N!}{k! \, (N - k)!} = \binom{N}{N - k}
\end{equation*}
that counts the number of possible ways to choose $k$ objects out of a set of $N \geq k$ total objects.
I also anticipate that you have previously seen \href{https://en.wikipedia.org/wiki/Gaussian_integral}{gaussian integrals},
\begin{equation*}
  \int_{-\infty}^{\infty} e^{-a (x + b)^2} \; dx = \sqrt{\frac{\pi}{a}}.
\end{equation*}
% ------------------------------------------------------------------



% ------------------------------------------------------------------
\subsection*{Assessment and academic integrity}
The assessment workload has also been kept the same as last year, though weights have been distributed more evenly across the term to accommodate current circumstances and avoid a high-stakes final assessment.
Deadlines for in-term assignments have been coordinated within the Department to minimize pile-up.
MATH327 has been allocated the following deadlines at 23:59 on \textbf{Tuesdays}:
\begin{description}
  \item[20\%] A computer-based project divided into two equally weighted parts, the first due \textbf{Tuesday, 2 March} and the second due \textbf{Tuesday, 13 April}
  \item[15\%] A homework assignment available by Monday, 22 February and due \textbf{Tuesday, 16 March}
  \item[15\%] A homework assignment available by Monday, 19 April and due \textbf{Tuesday, 27 April}
  \item[50\%] A final assessment to be centrally scheduled within the period 17 May through 4 June
\end{description}
Since the second part of the computer project is due shortly after the term break, feedback for the first part will be provided before the break, and office hours will continue during the break in case any assistance is needed with this project.

Following the University's \href{https://www.liverpool.ac.uk/media/livacuk/tqsd/code-of-practice-on-assessment/code_of_practice_on_assessment.pdf}{Code of Practice on Assessment}, late submissions completed within five days (120 hours) after the submission deadline will have 5\% of the total marks deducted for each 24-hour post-deadline period.
Submissions more than five days late will be awarded zero marks, though I will still endeavour to provide feedback on them.
I will aim to return feedback and solutions 7--10 days after the deadline.

Because the computer-based project will be done remotely, rather than in a computer lab on campus, you are free to complete it using the programming language of your choice.
Instruction on the necessary programming concepts will be provided using \href{https://www.python.org}{Python}, the free programming language recommended for the project.
If you have difficulty setting up Python on your device, you can run it for free online at \href{https://repl.it/languages/python3}{repl.it}. % onlinegdb.com has numpy but not matplotlib
(Make sure to keep a copy of the code to submit for assessment!)
Alternative languages could include \href{https://en.wikipedia.org/wiki/C_(programming_language)}{C}, \href{https://www.r-project.org}{R}, or even \href{https://matlab.mathworks.com}{MATLAB} (through the University's site license).
Maple may struggle to handle parts of the project.

I encourage you to discuss the in-term assignments with each other, since discussing and debating concepts and procedures is a very effective way to learn.
The examination must be done on your own, and your submissions for all assignments must be your own work representing your own understanding.
Clear and neat presentations of your workings and the logic behind them will contribute to your mark.
It is unacceptable to copy solutions in part or in whole from other students (current or prior) or from other sources (commercial or otherwise).
Should you make use of resources beyond the module materials, these must be explicitly referenced in your work.

By now you should have successfully passed the Academic Integrity Tutorial and Quiz to affirm that you have read and understood the Academic Integrity Policy detailed in Appendix L of the Code of Practice on Assessment.
If you have any questions about what is or is not acceptable under this policy, please ask me or Departmental Assessment Officer Kamila Zychaluk.
In all cases, the spirit of the Academic Integrity Policy should take precedence over legalistic convolutions of the text.
% ------------------------------------------------------------------



% ------------------------------------------------------------------
\subsection*{Additional resources}
The lists of further reading below provide some optional additional resources that may be helpful.
Hyperlinks lead either to the resource itself or to the corresponding record page in our library.

\noindent\textbf{Books and lecture notes at roughly the level of this module:} \\[-24 pt]
\begin{enumerate}
  \item David Tong, \href{https://www.damtp.cam.ac.uk/user/tong/statphys.html}{\textit{Lectures on Statistical Physics}} (2012), \\ www.damtp.cam.ac.uk/user/tong/statphys.html
  \item Daniel V.~Schroeder, \textit{An Introduction to Thermal Physics} (first edition, 2000)
  \item C.~Kittel and H.~Kroemer, \textit{Thermal Physics} (second edition, 1980)
  \item F.~Reif, \textit{Fundamentals of Statistical and Thermal Physics} (first edition, 1965)
\end{enumerate}

\noindent\textbf{More advanced and more specialized books}, which may be useful to consult concerning specific questions or topics: \\[-24 pt]
\begin{enumerate}
  \setcounter{enumi}{4}
  \item R.~K.~Pathria, \textit{Statistical Mechanics} (second edition, 1996)
  \item Sidney Redner, \textit{A Guide to First-Passage Processes} (first edition, 2007)
  \item Pavel L.~Krapivsky, Sidney Redner and Eli Ben-Naim, \textit{A Kinetic View of Statistical Physics} (first edition, 2010)
  \item Kerson Huang, \textit{Statistical Mechanics} (second edition, 1987)
  \item Weinan E, Tiejun Li and Eric Vanden-Eijnden, \textit{Applied Stochastic Analysis} (first edition, 2019)
  \item Michael Plischke and Birger Bergersen, \textit{Equilibrium Statistical Physics} (third edition, 2005)
  \item L.~D.~Landau and E.~M.~Lifshitz, \textit{Statistical Physics, Part 1} (third edition, 1980)
\end{enumerate}

\newpage % WARNING: FORMATTING BY HAND
\noindent\textbf{A general book about learning}, which (among other strategies) emphasizes the value of retrieval practice compared to re-reading lecture notes or re-watching videos: \\[-24 pt]
\begin{enumerate}
  \setcounter{enumi}{11}
  \item Peter C.~Brown, Henry L.~Roediger III and Mark A.~McDaniel, \textit{Make it Stick: The Science of Successful Learning} (first edition, 2014) \\
        A \href{https://www.youtube.com/watch?v=MfylloWuuZU}{short summary video} is also available
\end{enumerate}

\noindent\textbf{Maple and MATLAB resources:} \\[-24 pt]
\begin{enumerate}
  \setcounter{enumi}{12}
  \item Ian Thompson, \href{https://library.liv.ac.uk/record=b4395758~S8}{\textit{Understanding Maple}} (ebook edition, 2017) \\
        Related videos: \href{https://stream.liv.ac.uk/4k67bdzt}{\textit{Running Maple for the first time}} (stream.liv.ac.uk/4k67bdzt) \\
        \textcolor{white}{Related videos:} \href{https://stream.liv.ac.uk/7pjge23a}{\textit{Configuring Maple}} (stream.liv.ac.uk/7pjge23a)
  \item Stormy Attaway, \textit{MATLAB: A Practical Introduction to Programming and Problem Solving} (third edition, 2013)
  \item B.~Barnes and G.~R.~Fulford, \textit{Mathematical Modelling with Case Studies: Using Maple and MATLAB} (third edition, 2014)
\end{enumerate}

Finally, there is a vast constellation of purely online resources, such as \href{https://en.wikipedia.org/wiki/Statistical_physics}{Wikipedia}.
These are often fine places to \emph{start} learning about a subject, but may be terrible places to \emph{stop}.
% ------------------------------------------------------------------
